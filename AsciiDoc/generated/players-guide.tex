\section{}

Updated May 1, 2025\\
\emph{Highlighted Portions} are scheduled for significant
modification.\\
Sign up to receive email updates at
\url{https://wyrmworkspublishing.com/hoard}\\
Follow and Back the Kickstarter to make this a reality at
\url{https://www.kickstarter.com/projects/wyrmworkspublishing/free5e-a-free-open-source-dungeons-and-dragons-alternative?ref=dmx0j9}

\section{Introduction}\label{_introduction}

Welcome to the Free5e Player's Guide, brought to you by Wyrmworks
Publishing. Our mission is to enhance inclusivity and accessibility in
tabletop roleplaying. This guide is designed to help you create unique
characters, embark on exciting adventures, and explore worlds where
everyone can play, create, and share.

\section{How to Play}\label{_how_to_play}

Playing a tabletop roleplaying game (TTRPG) is all about imagining a
character in a fantasy world and using dice to determine what happens.
Here's how it works:

\begin{description}
\item[Choose a Game Master (GM).]
One player is the GM, who creates the story, controls the world, and
describes everything and everyone around you.
\item[Create Your Character.]
Choose an ancestry, class, and background for your character. This is
your hero in the game world!
\item[Playing an Encounter]
On your turn, you describe what your character does. Do you swing your
sword, cast a spell, or talk to someone? To see whether you succeed, you
roll a 20-sided die (d20) and add your ability score modifier and,
maybe, a proficiency bonus. Higher rolls are better!
\end{description}

Example of play

\textbf{Grey (GM):} The village elder has asked you to investigate a
strange ruin in the nearby forest. After hours of travel, you arrive at
a crumbling stone structure covered in moss. Inside, the air is damp,
and faint echoes of scratching come from within the stony depths.. What
do you do?

\textbf{Susan (playing Sylvana, a halfling Bard):} I step in quietly and
take a closer look at the walls. Do I recognize any symbols or writing?

\textbf{GM:} Make an Investigation check, using Intelligence.

\textbf{Susan:} (Rolls d20) That's a 12, plus 1 for Intelligence and 2
from my Investigation proficiency, so 15 total.

\textbf{GM:} You recognize some faded symbols of an ancient order
dedicated to protecting the forest. This was likely a Primal temple.

\textbf{Owen (playing Osmus, a human Ranger):} I listen closely to
pinpoint where the scratching is coming from. Perception check?

\textbf{GM:} Go ahead.

\textbf{Owen:} (Rolls d20) I got a 10, plus 4 for Wisdom is 14.

\textbf{GM:} The scratching is coming from behind a door at the far end
of a corridor leading inside the temple.

\textbf{Sylvana:} I cautiously open the door and peek inside.

\textbf{GM:} The door creaks open, revealing dim light glinting off
something large in the shadows. Its long, slimy tentacles sway as it
shifts, and you hear its beak click. It hisses, sensing you. Perched on
the ceiling, it stares down.

\textbf{Osmus:} What is that thing? It looks dangerous!

\textbf{GM:} It's certainly not friendly. Roll Initiative! (Rolls d20)
With its Dexterity bonus, it gets a 9.

\textbf{Osmus:} (Rolls d20) 9, plus 3 for Dexterity is 12.

\textbf{Sylvana:} (Rolls d20) 15! I'm first! I try to confuse it with a
quick spell. I cast Vicious Mockery, shouting, `You look like something
the forest spit out!' It needs to make a Wisdom saving throw.

\textbf{GM:} (Rolls for the creature) That's a 6. It fails.

\textbf{Sylvana:} (Rolls 1d4) It takes 3 psychic damage and has
disadvantage on its next attack!

\textbf{GM:} The creature seems momentarily stunned, its many eyes
narrowing as it hisses. Osmus, your turn!

\textbf{Osmus:} I fire an arrow at it! (Rolls d20) That's a 17, plus 5
to hit. Does that hit?

\textbf{GM:} Yes, that hits. Roll for damage.

\textbf{Osmus:} (Rolls 1d8) That's a 2, but I add my Dexterity so that's
a total 5 damage!

\textbf{GM:} The arrow strikes true, but the creature's tough hide
absorbs some of the blow. It lunges with its tentacles!

\textbf{GM (as the creature):} (Rolls to attack Osmus with the
tentacles) That's a 22 to hit versus your Armor Class.

\textbf{Osmus:} Ouch, I've only got 15 so that hits!

\textbf{GM:} (Rolls for damage) You take 10 bludgeoning damage, and I
need you to make a Strength saving throw to avoid being pulled in.

\textbf{Osmus:} (Rolls d20) That's a 14, plus 3 for Strength, so 17.

\textbf{GM:} You hold your ground, but the creature's tentacles are
still trying to wrap around you. It'll try again. What's your next move?

\textbf{Sylvana:} I step back and cast Command on the creature and shout
``Flee!'' to force it to flee! It needs to make a Wisdom saving throw,
DC 13.

\textbf{GM:} (Rolls for the creature) That's a 9. It fails!

\textbf{Sylvana:} It must use its reaction to move as far away as
possible!

\textbf{GM:} The creature screeches in pain, skittering across the
ceiling to the far corner, giving you space. It looks weakened but still
dangerous.

\textbf{Osmus:} Let's finish this!

The battle continues, with the party using teamwork and clever spells to
face down the fearsome creature!

In every game, describe your actions, roll dice to succeed, and react to
the unfolding story. The rules guide you, but the fun comes from your
shared stories. Let your imagination soar!

\section{Creating a Character}\label{_creating_a_character}

As a player, begin by creating a character on the character sheet in the
back of this book or a color-coded one at {[}LINK{]}. If you're using a
paper copy, we recommend writing in an erasable medium like pencil.

\begin{enumerate}
\def\labelenumi{\arabic{enumi}.}
\item
  Think of a fantasy character concept that you'd like to play
\item
  Determine ability scores
\item
  Choose your class
\item
  Choose your ancestry
\item
  Choose your heritage
\item
  Choose your background
\item
  Choose your starting equipment
\item
  Choose starting spells if applicable
\item
  Add details like appearance, personality, etc. Consider drawing a
  picture of your character. It doesn't have to be fancy! It's just for
  your friends! (If you'd like to commission a professional portrait of
  your character, check the credits of this book for some great artists
  who love illustrating characters!)
\end{enumerate}

\subsection{Determine Ability Scores}\label{_determine_ability_scores}

Use the following scores: 15, 14, 13, 12, 10, 8. Assign each of these
numbers to one of your character's six abilities: Strength, Dexterity,
Constitution, Intelligence, Wisdom, and Charisma. Then add 3 points to
the character's abilities, no more than two to any single ability.

\subsection{Variant: Ability Score Point
Buy}\label{_variant_ability_score_point_buy}

You have 27 points to spend on ability scores. The cost of each score is
shown on the Ability Score Point Cost table. Then add 3 points to the
character's abilities, no more than two to any single ability.

\begin{longtable}[]{@{}
  >{\raggedright\arraybackslash}p{(\linewidth - 2\tabcolsep) * \real{0.5000}}
  >{\raggedright\arraybackslash}p{(\linewidth - 2\tabcolsep) * \real{0.5000}}@{}}
\caption{Ability Score Point Cost}\tabularnewline
\toprule\noalign{}
\begin{minipage}[b]{\linewidth}\raggedright
Score
\end{minipage} & \begin{minipage}[b]{\linewidth}\raggedright
Cost
\end{minipage} \\
\midrule\noalign{}
\endfirsthead
\toprule\noalign{}
\begin{minipage}[b]{\linewidth}\raggedright
Score
\end{minipage} & \begin{minipage}[b]{\linewidth}\raggedright
Cost
\end{minipage} \\
\midrule\noalign{}
\endhead
\bottomrule\noalign{}
\endlastfoot
8 & 0 \\
9 & 1 \\
10 & 2 \\
11 & 3 \\
12 & 4 \\
13 & 5 \\
14 & 7 \\
15 & 9 \\
\end{longtable}

\subsection{Ability Modifiers}\label{_ability_modifiers}

Your final ability scores determine your modifiers. Find your modifier
by subtracting 10 from the ability score and dividing by 2 (round down).

\section{Classes}\label{_classes}

Class Name Changes

Some class names in Free5e have been updated to remove outdated or
problematic references while staying true to their themes. They use the
same game mechanics as the original classes.

\begin{description}
\item[Barbarian → Dreadnought]
Replacing a derogatory cultural slur
\item[Druid → Primal]
Removed inaccurate and appropriated portrayal of real-world religion
\item[Monk → Adept]
Expanding the concept without appropriating a cultural tradition
\item[Paladin → Vanguard]
Removing the association with a real-world religious conflict\}
\end{description}

\subsection{Dreadnought}\label{dreadnought-class}

Unstoppable in battle, Dreadnoughts draw strength from raw emotion,
turning fury and resilience into unmatched combat prowess.

\subsubsection{Class Features}\label{_class_features}

As a Dreadnought, you gain the following class features.

\paragraph{Hit Points}\label{_hit_points}

\begin{longtable}[]{@{}
  >{\raggedright\arraybackslash}p{(\linewidth - 2\tabcolsep) * \real{0.1500}}
  >{\raggedright\arraybackslash}p{(\linewidth - 2\tabcolsep) * \real{0.8500}}@{}}
\toprule\noalign{}
\endhead
\bottomrule\noalign{}
\endlastfoot
Hit Dice & 1d12 per Dreadnought level \\
Hit Points at 1st Level & 12 + your Constitution modifier \\
Hit Points at Higher Levels & 1d12 (or 7) + your Constitution modifier
per Dreadnought level after 1st \\
\end{longtable}

\paragraph{Proficiencies}\label{_proficiencies}

\begin{longtable}[]{@{}
  >{\raggedright\arraybackslash}p{(\linewidth - 2\tabcolsep) * \real{0.1500}}
  >{\raggedright\arraybackslash}p{(\linewidth - 2\tabcolsep) * \real{0.8500}}@{}}
\toprule\noalign{}
\endhead
\bottomrule\noalign{}
\endlastfoot
Armor & Light armor, medium armor, shields \\
Weapons & Simple weapons, martial weapons \\
Tools & None \\
Saving Throws & Strength, Constitution \\
Skills & Choose two from Animal Handling, Athletics, Intimidation,
Nature, Perception, and Survival \\
\end{longtable}

\paragraph{Equipment}\label{_equipment}

You start with the following equipment, in addition to the equipment
granted by your background:

\begin{itemize}
\item
  (a) a greataxe or (b) any martial melee weapon
\item
  (a) two handaxes or (b) any simple weapon
\item
  An explorer's pack and four javelins
\end{itemize}

\begin{longtable}[]{@{}
  >{\raggedright\arraybackslash}p{(\linewidth - 8\tabcolsep) * \real{0.1111}}
  >{\raggedright\arraybackslash}p{(\linewidth - 8\tabcolsep) * \real{0.2222}}
  >{\raggedright\arraybackslash}p{(\linewidth - 8\tabcolsep) * \real{0.4444}}
  >{\raggedright\arraybackslash}p{(\linewidth - 8\tabcolsep) * \real{0.1111}}
  >{\raggedright\arraybackslash}p{(\linewidth - 8\tabcolsep) * \real{0.1111}}@{}}
\caption{The Dreadnought
(table)}\label{dreadnought-table}\tabularnewline
\toprule\noalign{}
\begin{minipage}[b]{\linewidth}\raggedright
Level
\end{minipage} & \begin{minipage}[b]{\linewidth}\raggedright
Proficiency Bonus
\end{minipage} & \begin{minipage}[b]{\linewidth}\raggedright
Features
\end{minipage} & \begin{minipage}[b]{\linewidth}\centering
Rages
\end{minipage} & \begin{minipage}[b]{\linewidth}\raggedright
Rage Damage
\end{minipage} \\
\midrule\noalign{}
\endfirsthead
\toprule\noalign{}
\begin{minipage}[b]{\linewidth}\raggedright
Level
\end{minipage} & \begin{minipage}[b]{\linewidth}\raggedright
Proficiency Bonus
\end{minipage} & \begin{minipage}[b]{\linewidth}\raggedright
Features
\end{minipage} & \begin{minipage}[b]{\linewidth}\centering
Rages
\end{minipage} & \begin{minipage}[b]{\linewidth}\raggedright
Rage Damage
\end{minipage} \\
\midrule\noalign{}
\endhead
\bottomrule\noalign{}
\endlastfoot
1st & +2 & \hyperref[dreadnought-feature-rage]{Rage},
\hyperref[dreadnought-feature-unarmored-defense]{Unarmored Defense} & 2
& +2 \\
2nd & +2 & \hyperref[dreadnought-feature-reckless-attack]{Reckless
Attack}, \hyperref[dreadnought-feature-danger-sense]{Danger Sense} & 2 &
+2 \\
3rd & +2 & \hyperref[dreadnought-feature-path]{Dreadnought Path} & 3 &
+2 \\
4th & +2 & \hyperref[dreadnought-feature-asi]{Ability Score Improvement}
& 3 & +2 \\
5th & +3 & \hyperref[dreadnought-feature-extra-attack]{Extra Attack},
\hyperref[dreadnought-feature-fast-movement]{Fast Movement} & 3 & +2 \\
6th & +3 & Path feature & 4 & +2 \\
7th & +3 & \hyperref[dreadnought-feature-feral-instinct]{Feral Instinct}
& 4 & +2 \\
8th & +3 & \hyperref[dreadnought-feature-asi]{Ability Score Improvement}
& 4 & +2 \\
9th & +4 & \hyperref[dreadnought-feature-brutal-critical]{Brutal
Critical} (1 die) & 4 & +3 \\
10th & +4 & Path feature & 4 & +3 \\
11th & +4 & \hyperref[dreadnought-feature-relentless-rage]{Relentless
Rage} & 4 & +3 \\
12th & +4 & \hyperref[dreadnought-feature-asi]{Ability Score
Improvement} & 5 & +3 \\
13th & +5 & \hyperref[dreadnought-feature-brutal-critical]{Brutal
Critical} (2 dice) & 5 & +3 \\
14th & +5 & Path feature & 5 & +3 \\
15th & +5 & \hyperref[dreadnought-feature-persistent-rage]{Persistent
Rage} & 5 & +3 \\
16th & +5 & \hyperref[dreadnought-feature-asi]{Ability Score
Improvement} & 5 & +4 \\
17th & +6 & \hyperref[dreadnought-feature-brutal-critical]{Brutal
Critical} (3 dice) & 6 & +4 \\
18th & +6 & \hyperref[dreadnought-feature-indomitable-might]{Indomitable
Might} & 6 & +4 \\
19th & +6 & \hyperref[dreadnought-feature-asi]{Ability Score
Improvement} & 6 & +4 \\
20th & +6 & \hyperref[dreadnought-feature-primal-champion]{Primal
Champion} & Unlimited & +4 \\
\end{longtable}

\subsubsection{Rage}\label{dreadnought-feature-rage}

In battle, you fight with primal ferocity. On your turn, you can enter a
rage as a bonus action. While raging, you gain the following benefits if
you aren't wearing heavy armor:

\begin{itemize}
\item
  You have advantage on Strength checks and Strength saving throws.
\item
  When you make a melee weapon attack using Strength, you gain a bonus
  to the damage roll that increases as you gain levels as a Dreadnought,
  as shown in the Rage Damage column of the
  \hyperref[dreadnought-table]{Dreadnought table}.
\item
  You have resistance to bludgeoning, piercing, and slashing damage.
\end{itemize}

If you are able to cast spells, you can't cast them or concentrate on
them while raging. Your rage lasts for 1 minute. It ends early if you
are knocked unconscious or if your turn ends and you haven't attacked a
hostile creature since your last turn or taken damage since then. You
can also end your rage on your turn as a bonus action. Once you have
raged the number of times shown for your Dreadnought level in the Rages
column of the \hyperref[dreadnought-table]{Dreadnought table}, you must
finish a long rest before you can rage again.

\subsubsection{Unarmored
Defense}\label{dreadnought-feature-unarmored-defense}

While you are not wearing any armor, your Armor Class equals 10 + your
Dexterity modifier + your Constitution modifier. You can use a shield
and still gain this benefit.

\subsubsection{Reckless
Attack}\label{dreadnought-feature-reckless-attack}

Starting at 2nd level, you can throw aside all concern for defense to
attack with fierce desperation. When you make your first attack on your
turn, you can decide to attack recklessly. Doing so gives you advantage
on melee weapon attack rolls using Strength during this turn, but attack
rolls against you have advantage until your next turn.

\subsubsection{Danger Sense}\label{dreadnought-feature-danger-sense}

At 2nd level, you gain an uncanny sense of when things nearby aren't as
they should be, giving you an edge when you dodge away from danger. You
have advantage on Dexterity saving throws against effects that you can
see, such as traps and spells. To gain this benefit, you can't be
blinded, deafened, or incapacitated.

\subsubsection{Dreadnought Path}\label{dreadnought-feature-path}

At 3rd level, you choose a \hyperref[dreadnought-subclasses]{path} that
shapes the nature of your rage. Your choice grants you features at 3rd
level and again at 6th, 10th, and 14th levels.

\subsubsection{Ability Score Improvement}\label{dreadnought-feature-asi}

When you reach 4th level, and again at 8th, 12th, 16th, and 19th level,
you can increase one ability score of your choice by 2, or you can
increase two ability scores of your choice by 1. As normal, you can't
increase an ability score above 20 using this feature.

\subsubsection{Extra Attack}\label{dreadnought-feature-extra-attack}

Beginning at 5th level, you can attack twice, instead of once, whenever
you take the Attack action on your turn.

\subsubsection{Fast Movement}\label{dreadnought-feature-fast-movement}

Starting at 5th level, your speed increases by 10 feet while you aren't
wearing heavy armor.

\subsubsection{Feral Instinct}\label{dreadnought-feature-feral-instinct}

By 7th level, your instincts are so honed that you have advantage on
initiative rolls. Additionally, if you are surprised at the beginning of
combat and aren't incapacitated, you can act normally on your first
turn, but only if you enter your rage before doing anything else on that
turn.

\subsubsection{Brutal
Critical}\label{dreadnought-feature-brutal-critical}

Beginning at 9th level, you can roll one additional weapon damage die
when determining the extra damage for a critical hit with a melee
attack. This increases to two additional dice at 13th level and three
additional dice at 17th level.

\subsubsection{Relentless
Rage}\label{dreadnought-feature-relentless-rage}

Starting at 11th level, your rage can keep you fighting despite grievous
wounds. If you drop to 0 hit points while you're raging and don't die
outright, you can make a DC 10 Constitution saving throw. If you
succeed, you drop to 1 hit point instead. Each time you use this feature
after the first, the DC increases by 5. When you finish a short or long
rest, the DC resets to 10.

\subsubsection{Persistent
Rage}\label{dreadnought-feature-persistent-rage}

Beginning at 15th level, your rage is so fierce that it ends early only
if you fall unconscious or if you choose to end it.

\subsubsection{Indomitable
Might}\label{dreadnought-feature-indomitable-might}

Beginning at 18th level, if your total for a Strength check is less than
your Strength score, you can use that score in place of the total.

\subsubsection{Primal
Champion}\label{dreadnought-feature-primal-champion}

At 20th level, you embody the power of the wilds. Your Strength and
Constitution scores increase by 4. Your maximum for those scores is now
24.

\subsubsection{Dreadnought Paths}\label{dreadnought-subclasses}

\paragraph{Path of the Berserker}\label{dreadnought-subclasse-berserker}

For some Dreadnoughts, rage is a means to an end--- that end being
violence. The Path of the Berserker is a path of untrammeled fury, slick
with blood. As you enter the berserker's rage, you thrill in the chaos
of battle, heedless of your own health or well-being.

\subparagraph{Frenzy}\label{dreadnought-subclasse-berserker-feature-frenzy}

Starting when you choose this path at 3rd level, you can go into a
frenzy when you rage. If you do so, for the duration of your rage you
can make a single melee weapon attack as a bonus action on each of your
turns after this one. When your rage ends, you suffer one level of
exhaustion (as described in appendix A).

\subparagraph{Mindless
Rage}\label{dreadnought-subclasse-berserker-feature-mindless-rage}

Beginning at 6th level, you can't be charmed or frightened while raging.
If you are charmed or frightened when you enter your rage, the effect is
suspended for the duration of the rage.

\subparagraph{Intimidating
Presence}\label{dreadnought-subclasse-berserker-feature-intimidating-resence}

Beginning at 10th level, you can use your action to frighten someone
with your menacing presence. When you do so, choose one creature that
you can see within 30 feet of you. If the creature can see or hear you,
it must succeed on a Wisdom saving throw (DC equal to 8 + your
proficiency bonus + your Charisma modifier) or be frightened of you
until the end of your next turn. On subsequent turns, you can use your
action to extend the duration of this effect on the frightened creature
until the end of your next turn. This effect ends if the creature ends
its turn out of line of sight or more than 60 feet away from you. If the
creature succeeds on its saving throw, you can't use this feature on
that creature again for 24 hours.

\subparagraph{Retaliation}\label{dreadnought-subclasse-berserker-feature-retaliation}

Starting at 14th level, when you take damage from a creature that is
within 5 feet of you, you can use your reaction to make a melee weapon
attack against that creature.

\subsection{Vanguard}\label{vanguard-class}

Vanguards channel divine strength, standing as unyielding champions of
causes greater than themselves.

\subsubsection{Class Features}\label{_class_features_2}

As a Vanguard, you gain the following class features.

\paragraph{Hit Points}\label{_hit_points_2}

\begin{longtable}[]{@{}
  >{\raggedright\arraybackslash}p{(\linewidth - 2\tabcolsep) * \real{0.1500}}
  >{\raggedright\arraybackslash}p{(\linewidth - 2\tabcolsep) * \real{0.8500}}@{}}
\toprule\noalign{}
\endhead
\bottomrule\noalign{}
\endlastfoot
Hit Dice & 1d10 per Vanguard level \\
Hit Points at 1st Level & 10 + your Constitution modifier \\
Hit Points at Higher Levels & 1d10 (or 6) + your Constitution modifier
per Vanguard level after 1st \\
\end{longtable}

\paragraph{Proficiencies}\label{_proficiencies_2}

\begin{longtable}[]{@{}
  >{\raggedright\arraybackslash}p{(\linewidth - 2\tabcolsep) * \real{0.1500}}
  >{\raggedright\arraybackslash}p{(\linewidth - 2\tabcolsep) * \real{0.8500}}@{}}
\toprule\noalign{}
\endhead
\bottomrule\noalign{}
\endlastfoot
Armor & All armor, shields \\
Weapons & Simple weapons, martial weapons \\
Tools & None \\
Saving Throws & Wisdom, Charisma \\
Skills & Choose two from Athletics, Insight, Intimidation, Medicine,
Persuasion, and Religion \\
\end{longtable}

\paragraph{Equipment}\label{_equipment_2}

You start with the following equipment, in addition to the equipment
granted by your background:

\begin{itemize}
\item
  (a) a martial weapon and a shield or (b) two martial weapons
\item
  (a) five javelins or (b) any simple melee weapon
\item
  (a) a priest's pack or (b) an explorer's pack
\item
  Chain mail and a holy symbol
\end{itemize}

\begin{longtable}[]{@{}
  >{\raggedright\arraybackslash}p{(\linewidth - 14\tabcolsep) * \real{0.0833}}
  >{\raggedright\arraybackslash}p{(\linewidth - 14\tabcolsep) * \real{0.1667}}
  >{\raggedright\arraybackslash}p{(\linewidth - 14\tabcolsep) * \real{0.3333}}
  >{\raggedright\arraybackslash}p{(\linewidth - 14\tabcolsep) * \real{0.0833}}
  >{\raggedright\arraybackslash}p{(\linewidth - 14\tabcolsep) * \real{0.0833}}
  >{\raggedright\arraybackslash}p{(\linewidth - 14\tabcolsep) * \real{0.0833}}
  >{\raggedright\arraybackslash}p{(\linewidth - 14\tabcolsep) * \real{0.0833}}
  >{\raggedright\arraybackslash}p{(\linewidth - 14\tabcolsep) * \real{0.0833}}@{}}
\caption{The Vanguard (table)}\label{vanguard-table}\tabularnewline
\toprule\noalign{}
\begin{minipage}[b]{\linewidth}\raggedright
Level
\end{minipage} & \begin{minipage}[b]{\linewidth}\raggedright
Proficiency Bonus
\end{minipage} & \begin{minipage}[b]{\linewidth}\raggedright
Features
\end{minipage} & \begin{minipage}[b]{\linewidth}\centering
1st
\end{minipage} & \begin{minipage}[b]{\linewidth}\centering
2nd
\end{minipage} & \begin{minipage}[b]{\linewidth}\centering
3rd
\end{minipage} & \begin{minipage}[b]{\linewidth}\centering
4th
\end{minipage} & \begin{minipage}[b]{\linewidth}\centering
5th
\end{minipage} \\
\midrule\noalign{}
\endfirsthead
\toprule\noalign{}
\begin{minipage}[b]{\linewidth}\raggedright
Level
\end{minipage} & \begin{minipage}[b]{\linewidth}\raggedright
Proficiency Bonus
\end{minipage} & \begin{minipage}[b]{\linewidth}\raggedright
Features
\end{minipage} & \begin{minipage}[b]{\linewidth}\centering
1st
\end{minipage} & \begin{minipage}[b]{\linewidth}\centering
2nd
\end{minipage} & \begin{minipage}[b]{\linewidth}\centering
3rd
\end{minipage} & \begin{minipage}[b]{\linewidth}\centering
4th
\end{minipage} & \begin{minipage}[b]{\linewidth}\centering
5th
\end{minipage} \\
\midrule\noalign{}
\endhead
\bottomrule\noalign{}
\endlastfoot
1st & +2 & \hyperref[vanguard-feature-divine-sense]{Divine Sense},
\hyperref[vanguard-feature-lay-on-hands]{Lay on Hands} & - & - & - & - &
- \\
2nd & +2 & \hyperref[vanguard-feature-fighting-style]{Fighting Style},
\hyperref[vanguard-feature-spellcasting]{Spellcasting},
\hyperref[vanguard-feature-divine-smite]{Divine Smite} & 2 & - & - & - &
- \\
3rd & +2 & \hyperref[vanguard-feature-divine-health]{Divine Health},
\hyperref[vanguard-feature-oath-of-duty]{Oath of Duty} & 3 & - & - & - &
- \\
4th & +2 & \hyperref[vanguard-feature-asi]{Ability Score Improvement} &
3 & - & - & - & - \\
5th & +3 & \hyperref[vanguard-feature-extra-attack]{Extra Attack} & 4 &
2 & - & - & - \\
6th & +3 & \hyperref[vanguard-feature-aura-of-protection]{Aura of
Protection} & 4 & 2 & - & - & - \\
7th & +3 & Oath of Duty feature & 4 & 3 & - & - & - \\
8th & +3 & \hyperref[vanguard-feature-asi]{Ability Score Improvement} &
4 & 3 & - & - & - \\
9th & +4 & - & 4 & 3 & 2 & - & - \\
10th & +4 & \hyperref[vanguard-feature-aura-of-courage]{Aura of Courage}
& 4 & 3 & 2 & - & - \\
11th & +4 & \hyperref[vanguard-feature-improved-divine-smite]{Improved
Divine Smite} & 4 & 3 & 3 & - & - \\
12th & +4 & \hyperref[vanguard-feature-asi]{Ability Score Improvement} &
4 & 3 & 3 & - & - \\
13th & +5 & - & 4 & 3 & 3 & 1 & - \\
14th & +5 & \hyperref[vanguard-feature-cleansing-touch]{Cleansing Touch}
& 4 & 3 & 3 & 1 & - \\
15th & +5 & Oath of Duty feature & 4 & 3 & 3 & 2 & - \\
16th & +5 & \hyperref[vanguard-feature-asi]{Ability Score Improvement} &
4 & 3 & 3 & 2 & - \\
17th & +6 & - & 4 & 3 & 3 & 3 & 1 \\
18th & +6 & Aura improvements & 4 & 3 & 3 & 3 & 1 \\
19th & +6 & \hyperref[vanguard-feature-asi]{Ability Score Improvement} &
4 & 3 & 3 & 3 & 2 \\
20th & +6 & Oath of Duty feature & 4 & 3 & 3 & 3 & 2 \\
\end{longtable}

\subsubsection{Divine Sense}\label{vanguard-feature-divine-sense}

The presence of strong evil registers on your senses like a noxious
odor, and powerful good rings like heavenly music in your ears. As an
action, you can open your awareness to detect such forces. Until the end
of your next turn, you know the location of any celestial, fiend, or
undead within 60 feet of you that is not behind total cover. You know
the type (celestial, fiend, or undead) of any being whose presence you
sense, but not its identity (the vampire Count Strahd von Zarovich, for
instance). Within the same radius, you also detect the presence of any
place or object that has been consecrated or desecrated, as with the
Hallow spell. You can use this feature a number of times equal to 1 +
your Charisma modifier. When you finish a long rest, you regain all
expended uses.

\subsubsection{Lay on Hands}\label{vanguard-feature-lay-on-hands}

Your blessed touch can heal wounds. You have a pool of healing power
that replenishes when you take a long rest. With that pool, you can
restore a total number of hit points equal to your Vanguard level × 5.

As an action, you can touch a creature and draw power from the pool to
restore a number of hit points to that creature, up to the maximum
amount remaining in your pool.

Alternatively, you can expend 5 hit points from your pool of healing to
cure the target of one disease or neutralize one poison affecting it.
You can cure multiple diseases and neutralize multiple poisons with a
single use of Lay on Hands, expending hit points separately for each
one.

This feature has no effect on undead and constructs.

\subsubsection{Fighting Style}\label{vanguard-feature-fighting-style}

At 2nd level, you adopt a style of fighting as your specialty. Choose
one of the following options. You can't take a Fighting Style option
more than once, even if you later get to choose again.

\begin{description}
\item[Defense]
While you are wearing armor, you gain a +1 bonus to AC.
\end{description}

\begin{description}
\item[Dueling]
When you are wielding a melee weapon in one hand and no other weapons,
you gain a +2 bonus to damage rolls with that weapon.
\end{description}

\begin{description}
\item[Great Weapon Fighting]
When you roll a 1 or 2 on a damage die for an attack you make with a
melee weapon that you are wielding with two hands, you can reroll the
die and must use the new roll. The weapon must have the two-handed or
versatile property for you to gain this benefit.
\end{description}

\begin{description}
\item[Protection]
When a creature you can see attacks a target other than you that is
within 5 feet of you, you can use your reaction to impose disadvantage
on the attack roll. You must be wielding a shield.
\end{description}

\subsubsection{Spellcasting}\label{vanguard-feature-spellcasting}

By 2nd level, you have learned to draw on divine magic through
meditation and prayer to cast spells as a Cleric does.

\paragraph{Preparing and Casting
Spells}\label{_preparing_and_casting_spells}

The \hyperref[vanguard-table]{Vanguard table} shows how many spell slots
you have to cast your spells. To cast one of your Vanguard spells of 1st
level or higher, you must expend a slot of the spell's level or higher.
You regain all expended spell slots when you finish a long rest.

You prepare the list of Vanguard spells that are available for you to
cast, choosing from the Vanguard spell list. When you do so, choose a
number of Vanguard spells equal to your Charisma modifier + half your
Vanguard level, rounded down (minimum of one spell). The spells must be
of a level for which you have spell slots.

For example, if you are a 5th-level Vanguard, you have four 1st-level
and two 2nd-level spell slots. With a Charisma of 14, your list of
prepared spells can include four spells of 1st or 2nd level, in any
combination. If you prepare the 1st-level spell Cure Wounds, you can
cast it using a 1st-level or a 2nd-level slot. Casting the spell doesn't
remove it from your list of prepared spells.

You can change your list of prepared spells when you finish a long rest.
Preparing a new list of Vanguard spells requires time spent in prayer
and meditation: at least 1 minute per spell level for each spell on your
list.

\paragraph{Spellcasting Ability}\label{_spellcasting_ability}

Charisma is your spellcasting ability for your Vanguard spells, since
their power derives from the strength of your convictions. You use your
Charisma whenever a spell refers to your spellcasting ability. In
addition, you use your Charisma modifier when setting the saving throw
DC for a Vanguard spell you cast and when making an attack roll with
one.

\begin{longtable}[]{@{}
  >{\raggedright\arraybackslash}p{(\linewidth - 2\tabcolsep) * \real{0.1500}}
  >{\raggedright\arraybackslash}p{(\linewidth - 2\tabcolsep) * \real{0.8500}}@{}}
\toprule\noalign{}
\endhead
\bottomrule\noalign{}
\endlastfoot
Spell save DC & = 8 + your proficiency bonus + your Charisma modifier \\
Spell attack modifier & = your proficiency bonus + your Charisma
modifier \\
\end{longtable}

\paragraph{Spellcasting Focus}\label{_spellcasting_focus}

You can use a holy symbol as a spellcasting focus for your Vanguard
spells.

\subsubsection{Divine Smite}\label{vanguard-feature-divine-smite}

Starting at 2nd level, when you hit a creature with a melee weapon
attack, you can expend one spell slot to deal radiant damage to the
target, in addition to the weapon's damage. The extra damage is 2d8 for
a 1st-level spell slot, plus 1d8 for each spell level higher than 1st,
to a maximum of 5d8. The damage increases by 1d8 if the target is an
undead or a fiend.

\subsubsection{Divine Health}\label{vanguard-feature-divine-health}

By 3rd level, the divine magic flowing through you makes you immune to
disease.

\subsubsection{Oath of Duty}\label{vanguard-feature-oath-of-duty}

When you reach 3rd level, you swear the
\hyperref[vanguard-subclasses]{oath} that binds you as a Vanguard
forever. Up to this time you have been in a preparatory stage, committed
to the path but not yet sworn to it. Your choice grants you features at
3rd level and again at 7th, 15th, and 20th level. Those features include
oath spells and the Channel Divinity feature.

\paragraph{Oath Spells}\label{vanguard-feature-oath-of-duty-oath-spells}

Each oath has a list of associated spells. You gain access to these
spells at the levels specified in the oath description. Once you gain
access to an oath spell, you always have it prepared. Oath spells don't
count against the number of spells you can prepare each day. If you gain
an oath spell that doesn't appear on the Vanguard spell list, the spell
is nonetheless a Vanguard spell for you.

\paragraph{Channel
Divinity}\label{vanguard-feature-oath-of-duty-channel-divinity}

Your oath allows you to channel divine energy to fuel magical effects.
Each Channel Divinity option provided by your oath explains how to use
it. When you use your Channel Divinity, you choose which option to use.
You must then finish a short or long rest to use your Channel Divinity
again. Some Channel Divinity effects require saving throws. When you use
such an effect from this class, the DC equals your Vanguard spell save
DC.

\subsubsection{Ability Score Improvement}\label{vanguard-feature-asi}

When you reach 4th level, and again at 8th, 12th, 16th, and 19th level,
you can increase one ability score of your choice by 2, or you can
increase two ability scores of your choice by 1. As normal, you can't
increase an ability score above 20 using this feature.

\subsubsection{Extra Attack}\label{vanguard-feature-extra-attack}

Beginning at 5th level, you can attack twice, instead of once, whenever
you take the Attack action on your turn.

\subsubsection{Aura of
Protection}\label{vanguard-feature-aura-of-protection}

Starting at 6th level, whenever you or a friendly creature within 10
feet of you must make a saving throw, the creature gains a bonus to the
saving throw equal to your Charisma modifier (with a minimum bonus of
+1). You must be conscious to grant this bonus.

At 18th level, the range of this aura increases to 30 feet.

\subsubsection{Aura of Courage}\label{vanguard-feature-aura-of-courage}

Starting at 10th level, you and friendly creatures within 10 feet of you
can't be frightened while you are conscious.

At 18th level, the range of this aura increases to 30 feet.

\subsubsection{Improved Divine
Smite}\label{vanguard-feature-improved-divine-smite}

By 11th level, you are so suffused with righteous might that all your
melee weapon strikes carry divine power with them. Whenever you hit a
creature with a melee weapon, the creature takes an extra 1d8 radiant
damage. If you also use your Divine Smite with an attack, you add this
damage to the extra damage of your Divine Smite.

\subsubsection{Cleansing Touch}\label{vanguard-feature-cleansing-touch}

Beginning at 14th level, you can use your action to end one spell on
yourself or on one willing creature that you touch. You can use this
feature a number of times equal to your Charisma modifier (a minimum of
once). You regain expended uses when you finish a long rest.

\subsubsection{Oaths of Duty}\label{vanguard-subclasses}

Becoming a Vanguard involves taking vows that commit the Vanguard to the
cause of righteousness, an active path of fighting wickedness. The final
oath, taken when he or she reaches 3rd level, is the culmination of all
the Vanguard's training. Some characters with this class don't consider
themselves true Vanguards until they have reached 3rd level and made
this oath. For others, the actual swearing of the oath is a formality,
an official stamp on what has always been true in the Vanguard's heart.

\paragraph{Oath of Devotion}\label{vanguard-subclasse-oath-of-devotion}

The Oath of Devotion binds a Vanguard to the loftiest ideals of justice,
virtue, and order. Sometimes called cavaliers, white knights, or holy
warriors, the Vanguards meet the ideal of the knight in shining armor,
acting with honor in pursuit of justice and the greater good. They hold
themselves to the highest standards of conduct, and some, for better or
worse, hold the rest of the world to the same standards. Many who swear
this oath are devoted to gods of law and good and use their gods' tenets
as the measure of their devotion. They hold angels---the perfect
servants of good---as their ideals, and incorporate images of angelic
wings into their helmets or coats of arms.

\subparagraph{Tenets of Devotion}\label{_tenets_of_devotion}

Though the exact words and strictures of the Oath of Devotion vary,
Vanguards of this oath share these tenets.

\begin{longtable}[]{@{}
  >{\raggedright\arraybackslash}p{(\linewidth - 2\tabcolsep) * \real{0.1500}}
  >{\raggedright\arraybackslash}p{(\linewidth - 2\tabcolsep) * \real{0.8500}}@{}}
\toprule\noalign{}
\endhead
\bottomrule\noalign{}
\endlastfoot
Honesty. & Don't lie or cheat. Let your word be your promise. \\
Courage. & Never fear to act, though caution is wise. \\
Compassion. & Aid others, protect the weak, and punish those who
threaten them. Show mercy to your foes, but temper it with wisdom. \\
Honor. & Treat others with fairness, and let your honorable deeds be an
example to them. Do as much good as possible while causing the least
amount of harm. \\
Duty. & Be responsible for your actions and their consequences, protect
those entrusted to your care, and obey those who have just authority
over you. \\
\end{longtable}

\subparagraph{Oath Spells}\label{_oath_spells}

You gain oath spells at the Vanguard levels listed.

\begin{longtable}[]{@{}
  >{\raggedright\arraybackslash}p{(\linewidth - 2\tabcolsep) * \real{0.2000}}
  >{\raggedright\arraybackslash}p{(\linewidth - 2\tabcolsep) * \real{0.8000}}@{}}
\caption{Oath of Devotion Spells
(table)}\label{vanguard-oath-of-devotion-spells}\tabularnewline
\toprule\noalign{}
\begin{minipage}[b]{\linewidth}\raggedright
Level
\end{minipage} & \begin{minipage}[b]{\linewidth}\raggedright
Vanguard Spells
\end{minipage} \\
\midrule\noalign{}
\endfirsthead
\toprule\noalign{}
\begin{minipage}[b]{\linewidth}\raggedright
Level
\end{minipage} & \begin{minipage}[b]{\linewidth}\raggedright
Vanguard Spells
\end{minipage} \\
\midrule\noalign{}
\endhead
\bottomrule\noalign{}
\endlastfoot
3rd & Protection From Evil And Good, Sanctuary \\
5th & Lesser Restoration, Zone of Truth \\
9th & Beacon of Hope, Dispel Magic \\
13th & Freedom of Movement, Guardian of Faith \\
17th & Commune, Flame Strike \\
\end{longtable}

\subparagraph{Channel Divinity}\label{_channel_divinity}

When you take this oath at 3rd level, you gain the following two Channel
Divinity options.

\begin{description}
\item[Sacred Weapon.]
As an action, you can imbue one weapon that you are holding with
positive energy, using your Channel Divinity. For 1 minute, you add your
Charisma modifier to attack rolls made with that weapon (with a minimum
bonus of +1). The weapon also emits bright light in a 20-foot radius and
dim light 20 feet beyond that. If the weapon is not already magical, it
becomes magical for the duration.

You can end this effect on your turn as part of any other action. If you
are no longer holding or carrying this weapon, or if you fall
unconscious, this effect ends.
\item[Turn the Unholy.]
As an action, you present your holy symbol and speak a prayer censuring
fiends and undead, using your Channel Divinity. Each fiend or undead
that can see or hear you within 30 feet of you must make a Wisdom saving
throw. If the creature fails its saving throw, it is turned for 1 minute
or until it takes damage.

A turned creature must spend its turns trying to move as far away from
you as it can, and it can't willingly move to a space within 30 feet of
you. It also can't take reactions. For its action, it can use only the
Dash action or try to escape from an effect that prevents it from
moving. If there's nowhere to move, the creature can use the Dodge
action.
\end{description}

\subparagraph{Aura of Devotion}\label{_aura_of_devotion}

Starting at 7th level, you and friendly creatures within 10 feet of you
can't be charmed while you are conscious.

At 18th level, the range of this aura increases to 30 feet.

\subparagraph{Purity of Spirit}\label{_purity_of_spirit}

Beginning at 15th level, you are always under the effects of a
Protection from Evil and Good spell.

\subparagraph{Holy Nimbus}\label{_holy_nimbus}

At 20th level, as an action, you can emanate an aura of sunlight. For 1
minute, bright light shines from you in a 30-foot radius, and dim light
shines 30 feet beyond that. Whenever an enemy creature starts its turn
in the bright light, the creature takes 10 radiant damage.

In addition, for the duration, you have advantage on saving throws
against spells cast by fiends or undead.

Once you use this feature, you can't use it again until you finish a
long rest.

\subsubsection{Breaking Your Oath}\label{_breaking_your_oath}

A Vanguard tries to hold to the highest standards of conduct, but even
the most virtuous Vanguard is fallible. Sometimes the right path proves
too demanding, sometimes a situation calls for the lesser of two evils,
and sometimes the heat of emotion causes a Vanguard to transgress his or
her oath.

A Vanguard who has broken a vow typically seeks absolution from a Cleric
who shares his or her faith or from another Vanguard of the same order.
The Vanguard might spend an all- night vigil in prayer as a sign of
penitence, or undertake a fast or similar act of self-denial. After a
rite of confession and forgiveness, the Vanguard starts fresh.

If a Vanguard willfully violates his or her oath and shows no sign of
repentance, the consequences can be more serious. At the GM's
discretion, an impenitent Vanguard might be forced to abandon this class
and adopt another.

\subsection{Wizard}\label{_wizard}

Through study, discipline, and boundless curiosity, Wizards unravel the
secrets of magic, bending reality through sheer knowledge.

\subsubsection{Class Features}\label{_class_features_3}

As a Wizard, you gain the following class features.

\paragraph{Hit Points}\label{_hit_points_3}

\begin{longtable}[]{@{}
  >{\raggedright\arraybackslash}p{(\linewidth - 2\tabcolsep) * \real{0.1500}}
  >{\raggedright\arraybackslash}p{(\linewidth - 2\tabcolsep) * \real{0.8500}}@{}}
\toprule\noalign{}
\endhead
\bottomrule\noalign{}
\endlastfoot
Hit Dice & 1d6 per Wizard level \\
Hit Points at 1st Level & 6 + your Constitution modifier \\
Hit Points at Higher Levels & 1d6 (or 4) + your Constitution modifier
per Wizard level after 1st \\
\end{longtable}

\paragraph{Proficiencies}\label{_proficiencies_3}

\begin{longtable}[]{@{}
  >{\raggedright\arraybackslash}p{(\linewidth - 2\tabcolsep) * \real{0.1500}}
  >{\raggedright\arraybackslash}p{(\linewidth - 2\tabcolsep) * \real{0.8500}}@{}}
\toprule\noalign{}
\endhead
\bottomrule\noalign{}
\endlastfoot
Armor & None \\
Weapons & Daggers, darts, slings, quarterstaffs, light crossbows \\
Tools & None \\
Saving Throws & Intelligence, Wisdom \\
Skills & Choose two from Arcana, History, Insight, Investigation,
Medicine, and Religion \\
\end{longtable}

\paragraph{Equipment}\label{_equipment_3}

You start with the following equipment, in addition to the equipment
granted by your background:

\begin{itemize}
\item
  (a) a quarterstaff or (b) a dagger
\item
  (a) a component pouch or (b) an arcane focus
\item
  (a) a scholar's pack or (b) an explorer's pack
\item
  A spellbook
\end{itemize}

\begin{longtable}[]{@{}
  >{\raggedright\arraybackslash}p{(\linewidth - 24\tabcolsep) * \real{0.0750}}
  >{\raggedright\arraybackslash}p{(\linewidth - 24\tabcolsep) * \real{0.1500}}
  >{\raggedright\arraybackslash}p{(\linewidth - 24\tabcolsep) * \real{0.1250}}
  >{\raggedright\arraybackslash}p{(\linewidth - 24\tabcolsep) * \real{0.0500}}
  >{\raggedright\arraybackslash}p{(\linewidth - 24\tabcolsep) * \real{0.0500}}
  >{\raggedright\arraybackslash}p{(\linewidth - 24\tabcolsep) * \real{0.0500}}
  >{\raggedright\arraybackslash}p{(\linewidth - 24\tabcolsep) * \real{0.0500}}
  >{\raggedright\arraybackslash}p{(\linewidth - 24\tabcolsep) * \real{0.0500}}
  >{\raggedright\arraybackslash}p{(\linewidth - 24\tabcolsep) * \real{0.0500}}
  >{\raggedright\arraybackslash}p{(\linewidth - 24\tabcolsep) * \real{0.0500}}
  >{\raggedright\arraybackslash}p{(\linewidth - 24\tabcolsep) * \real{0.0500}}
  >{\raggedright\arraybackslash}p{(\linewidth - 24\tabcolsep) * \real{0.0500}}
  >{\raggedright\arraybackslash}p{(\linewidth - 24\tabcolsep) * \real{0.2000}}@{}}
\caption{The Wizard (table)}\label{wizard-table}\tabularnewline
\toprule\noalign{}
\begin{minipage}[b]{\linewidth}\raggedright
Level
\end{minipage} & \begin{minipage}[b]{\linewidth}\raggedright
Proficiency Bonus
\end{minipage} & \begin{minipage}[b]{\linewidth}\centering
Cantrips Known
\end{minipage} & \begin{minipage}[b]{\linewidth}\centering
1st
\end{minipage} & \begin{minipage}[b]{\linewidth}\centering
2nd
\end{minipage} & \begin{minipage}[b]{\linewidth}\centering
3rd
\end{minipage} & \begin{minipage}[b]{\linewidth}\centering
4th
\end{minipage} & \begin{minipage}[b]{\linewidth}\centering
5th
\end{minipage} & \begin{minipage}[b]{\linewidth}\centering
6th
\end{minipage} & \begin{minipage}[b]{\linewidth}\centering
7th
\end{minipage} & \begin{minipage}[b]{\linewidth}\centering
8th
\end{minipage} & \begin{minipage}[b]{\linewidth}\centering
9th
\end{minipage} & \begin{minipage}[b]{\linewidth}\raggedright
Features
\end{minipage} \\
\midrule\noalign{}
\endfirsthead
\toprule\noalign{}
\begin{minipage}[b]{\linewidth}\raggedright
Level
\end{minipage} & \begin{minipage}[b]{\linewidth}\raggedright
Proficiency Bonus
\end{minipage} & \begin{minipage}[b]{\linewidth}\centering
Cantrips Known
\end{minipage} & \begin{minipage}[b]{\linewidth}\centering
1st
\end{minipage} & \begin{minipage}[b]{\linewidth}\centering
2nd
\end{minipage} & \begin{minipage}[b]{\linewidth}\centering
3rd
\end{minipage} & \begin{minipage}[b]{\linewidth}\centering
4th
\end{minipage} & \begin{minipage}[b]{\linewidth}\centering
5th
\end{minipage} & \begin{minipage}[b]{\linewidth}\centering
6th
\end{minipage} & \begin{minipage}[b]{\linewidth}\centering
7th
\end{minipage} & \begin{minipage}[b]{\linewidth}\centering
8th
\end{minipage} & \begin{minipage}[b]{\linewidth}\centering
9th
\end{minipage} & \begin{minipage}[b]{\linewidth}\raggedright
Features
\end{minipage} \\
\midrule\noalign{}
\endhead
\bottomrule\noalign{}
\endlastfoot
1st & +2 & 3 & 2 & - & - & - & - & - & - & - & - &
\hyperref[wizard-feature-spellcasting]{Spellcasting},
\hyperref[wizard-feature-arcane-recovery]{Arcane Recovery} \\
2nd & +2 & 3 & 3 & - & - & - & - & - & - & - & - &
\hyperref[wizard-feature-arcane-tradition]{Arcane Tradition} \\
3rd & +2 & 3 & 4 & 2 & - & - & - & - & - & - & - & - \\
4th & +2 & 4 & 4 & 3 & - & - & - & - & - & - & - &
\hyperref[wizard-feature-asi]{Ability Score Improvement} \\
5th & +3 & 4 & 4 & 3 & 2 & - & - & - & - & - & - & - \\
6th & +3 & 4 & 4 & 3 & 3 & - & - & - & - & - & - & Arcane Tradition
feature \\
7th & +3 & 4 & 4 & 3 & 3 & 1 & - & - & - & - & - & - \\
8th & +3 & 4 & 4 & 3 & 3 & 2 & - & - & - & - & - &
\hyperref[wizard-feature-asi]{Ability Score Improvement} \\
9th & +4 & 4 & 4 & 3 & 3 & 3 & 1 & - & - & - & - & - \\
10th & +4 & 5 & 4 & 3 & 3 & 3 & 2 & - & - & - & - & Arcane Tradition
feature \\
11th & +4 & 5 & 4 & 3 & 3 & 3 & 2 & 1 & - & - & - & - \\
12th & +4 & 5 & 4 & 3 & 3 & 3 & 2 & 1 & - & - & - &
\hyperref[wizard-feature-asi]{Ability Score Improvement} \\
13th & +5 & 5 & 4 & 3 & 3 & 3 & 2 & 1 & 1 & - & - & - \\
14th & +5 & 5 & 4 & 3 & 3 & 3 & 2 & 1 & 1 & - & - & Arcane Tradition
feature \\
15th & +5 & 5 & 4 & 3 & 3 & 3 & 2 & 1 & 1 & 1 & - & - \\
16th & +5 & 5 & 4 & 3 & 3 & 3 & 2 & 1 & 1 & 1 & - &
\hyperref[wizard-feature-asi]{Ability Score Improvement} \\
17th & +6 & 5 & 4 & 3 & 3 & 3 & 2 & 1 & 1 & 1 & 1 & - \\
18th & +6 & 5 & 4 & 3 & 3 & 3 & 3 & 1 & 1 & 1 & 1 &
\hyperref[wizard-feature-spell-mastery]{Spell Mastery} \\
19th & +6 & 5 & 4 & 3 & 3 & 3 & 3 & 2 & 1 & 1 & 1 &
\hyperref[wizard-feature-asi]{Ability Score Improvement} \\
20th & +6 & 5 & 4 & 3 & 3 & 3 & 3 & 2 & 2 & 1 & 1 &
\hyperref[wizard-feature-signature-spells]{Signature Spells} \\
\end{longtable}

\subsubsection{Spellcasting}\label{wizard-feature-spellcasting}

As a student of arcane magic, you have a spellbook containing spells
that show the first glimmerings of your true power.

\paragraph{Cantrips}\label{_cantrips}

At 1st level, you know three cantrips of your choice from the Wizard
spell list. You learn additional Wizard cantrips of your choice at
higher levels, as shown in the Cantrips Known column of the
\hyperref[wizard-table]{Wizard table}.

\paragraph{Spellbook}\label{_spellbook}

At 1st level, you have a spellbook containing six 1st-level Wizard
spells of your choice. Your spellbook is the repository of the Wizard
spells you know, except your cantrips, which are fixed in your mind.

\paragraph{Your Spellbook}\label{_your_spellbook}

The spells you add to your spellbook reflect your arcane research and
intellectual breakthroughs about the multiverse. You might find other
spells during adventures, like a scroll in an evil Wizard's chest or a
dusty tome in an ancient library.

\begin{description}
\item[Copying a Spell into the Book.]
When you find a Wizard spell of 1st level or higher, you can add it to
your spellbook if it is of a spell level you can prepare and if you can
spare the time to decipher and copy it. Copying a spell into your
spellbook involves reproducing its basic form and deciphering its unique
notation. Practice until you understand the sounds and gestures, then
transcribe it using your notation. Each level takes 2 hours and costs 50
gp. This includes material components and fine inks for experimentation
and recording. Once mastered, you can prepare the spell like other
spells.
\item[Replacing the Book.]
You can copy a spell from your own spellbook into another book---for
example, if you want to make a backup copy of your spellbook. This is
just like copying a new spell into your spellbook, but faster and
easier, since you understand your own notation and already know how to
cast the spell. You need spend only 1 hour and 10 gp for each level of
the copied spell. If you lose your spellbook, you can use the same
procedure to transcribe the spells that you have prepared into a new
spellbook. Filling out the remainder of your spellbook requires you to
find new spells to do so, as normal. For this reason, many Wizards keep
backup spellbooks in a safe place.
\item[The Book's Appearance.]
Your spellbook is a unique compilation of spells, with its own
decorative flourishes and margin notes. It might be a plain, functional
leather volume that you received as a gift from your master, a finely
bound gilt-edged tome you found in an ancient library, or even a loose
collection of notes scrounged together after you lost your previous
spellbook in a mishap.
\end{description}

\paragraph{Preparing and Casting
Spells}\label{_preparing_and_casting_spells_2}

The \hyperref[wizard-table]{Wizard table} shows how many spell slots you
have to cast your spells of 1st level and higher. To cast one of these
spells, you must expend a slot of the spell's level or higher. You
regain all expended spell slots when you finish a long rest. You prepare
the list of Wizard spells that are available for you to cast. To do so,
choose a number of Wizard spells from your spellbook equal to your
Intelligence modifier + your Wizard level (minimum of one spell). The
spells must be of a level for which you have spell slots. For example,
if you're a 3rd-level Wizard, you have four 1st-level and two 2nd-level
spell slots. With an Intelligence of 16, your list of prepared spells
can include six spells of 1st or 2nd level, in any combination, chosen
from your spellbook. If you prepare the 1st-level spell Magic Missile,
you can cast it using a 1st-level or a 2nd-level slot. Casting the spell
doesn't remove it from your list of prepared spells. You can change your
list of prepared spells when you finish a long rest. Preparing a new
list of Wizard spells requires time spent studying your spellbook and
memorizing the incantations and gestures you must make to cast the
spell: at least 1 minute per spell level for each spell on your list.

\paragraph{Spellcasting Ability}\label{_spellcasting_ability_2}

Intelligence is your spellcasting ability for your Wizard spells, since
you learn your spells through dedicated study and memorization. You use
your Intelligence whenever a spell refers to your spellcasting ability.
In addition, you use your Intelligence modifier when setting the saving
throw DC for a Wizard spell you cast and when making an attack roll with
one.

\begin{longtable}[]{@{}
  >{\raggedright\arraybackslash}p{(\linewidth - 2\tabcolsep) * \real{0.1500}}
  >{\raggedright\arraybackslash}p{(\linewidth - 2\tabcolsep) * \real{0.8500}}@{}}
\toprule\noalign{}
\endhead
\bottomrule\noalign{}
\endlastfoot
Spell save DC & = 8 + your proficiency bonus + your Intelligence
modifier \\
Spell attack modifier & = your proficiency bonus + your Intelligence
modifier \\
\end{longtable}

\paragraph{Ritual Casting}\label{_ritual_casting}

You can cast a Wizard spell as a ritual if that spell has the ritual tag
and you have the spell in your spellbook. You don't need to have the
spell prepared.

\paragraph{Spellcasting Focus}\label{_spellcasting_focus_2}

You can use an arcane focus as a spellcasting focus for your Wizard
spells.

\paragraph{Learning Spells of 1st Level and
Higher}\label{_learning_spells_of_1st_level_and_higher}

Each time you gain a Wizard level, you can add two Wizard spells of your
choice to your spellbook for free. Each of these spells must be of a
level for which you have spell slots, as shown on the
\hyperref[wizard-table]{Wizard table}. On your adventures, you might
find other spells that you can add to your spellbook (see the ``Your
Spellbook'' sidebar).

\subsubsection{Arcane Recovery}\label{wizard-feature-arcane-recovery}

You have learned to regain some of your magical energy by studying your
spellbook. Once per day when you finish a short rest, you can choose
expended spell slots to recover. The spell slots can have a combined
level that is equal to or less than half your Wizard level (rounded up),
and none of the slots can be 6th level or higher. For example, if you're
a 4th-level Wizard, you can recover up to two levels worth of spell
slots. You can recover either a 2nd-level spell slot or two 1st-level
spell slots.

\subsubsection{Arcane Tradition}\label{wizard-feature-arcane-tradition}

When you reach 2nd level, you choose an
\hyperref[wizard-subclasses]{arcane tradition}. Your choice grants you
features at 2nd level and again at 6th, 10th, and 14th level.

\subsubsection{Ability Score Improvement}\label{wizard-feature-asi}

When you reach 4th level, and again at 8th, 12th, 16th, and 19th level,
you can increase one ability score of your choice by 2, or you can
increase two ability scores of your choice by 1. As normal, you can't
increase an ability score above 20 using this feature.

\subsubsection{Spell Mastery}\label{wizard-feature-spell-mastery}

At 18th level, you have achieved such mastery over certain spells that
you can cast them at will. Choose a 1st-level Wizard spell and a
2nd-level Wizard spell that are in your spellbook. You can cast those
spells at their lowest level without expending a spell slot when you
have them prepared. If you want to cast either spell at a higher level,
you must expend a spell slot as normal.

By spending 8 hours in study, you can exchange one or both of the spells
you chose for different spells of the same levels.

\subsubsection{Signature Spells}\label{wizard-feature-signature-spells}

When you reach 20th level, you gain mastery over two powerful spells and
can cast them with little effort. Choose two 3rd-level Wizard spells in
your spellbook as your signature spells. You always have these spells
prepared, they don't count against the number of spells you have
prepared, and you can cast each of them once at 3rd level without
expending a spell slot. When you do so, you can't do so again until you
finish a short or long rest. If you want to cast either spell at a
higher level, you must expend a spell slot as normal.

\subsubsection{Arcane Traditions}\label{wizard-subclasses}

The study of Wizardry, dating back to early magical discoveries, is
prevalent in fantasy gaming worlds with diverse magical traditions.

The most common arcane traditions revolve around the eight schools of
magic, cataloged by Wizards throughout history: Abjuration, Conjuration,
Divination, Enchantment, Evocation, Illusion, Necromancy and
Transmutation. These schools can be literal institutions, like the
School of Illusion, or academic departments with rival faculties. Even
Wizards who train apprentices use the school division as a learning
device, as each school requires mastery of different techniques.

\paragraph{Arcanist}\label{wizard-subclass-arcanist}

\subparagraph{Scholarly
Specialty}\label{wizard-subclass-arcanist-scholarly-speciality}

When you take this archetype at 2nd level, choose one classical school
of magic as your Scholarly Specialty: abjuration, conjuration,
divination, enchantment, evocation, illusion, necromancy, or
transmutation. The gold and time you must spend to copy spells from this
school into your spellbook is halved. If a feature refers to your chosen
school, it refers to the school selected in this feature.

\subparagraph{Esoteric
Talent}\label{wizard-subclass-arcanist-esoteric-talent}

Also at 2nd level, you gain one of the following benefits:

\begin{description}
\item[Bend Magic]
When you cast a wizard spell with an instantaneous duration that deals
damage to an area, you can choose a number of creatures in the area that
you can see equal to your Intelligence modifier (minimum one creature).
The chosen creatures take no damage from the spell.
\end{description}

\begin{description}
\item[Flash of Insight]
You can use a bonus action to roll a d20, record the result, and choose
a creature you can see within 30 feet. The next time that creature makes
an attack roll, ability check, or saving throw, it takes that d20 result
instead of rolling. If you use this feature again before you finish a
short rest, you must expend a spell slot of 1st-level or higher to do
so.
\end{description}

\begin{description}
\item[Quick Step]
After you cast a wizard spell of 1st-level or higher, you can
immediately move up to 15 feet without provoking opportunity attacks or
spending any of your normal movement.
\end{description}

\subparagraph{Refined
Learning}\label{wizard-subclass-arcanist-refined-learning}

At 6th level, choose one of the following benefits:

\begin{description}
\item[Arcane Armor]
When you cast a wizard spell of 1st-level or higher, you store some of
its magic to protect yourself, gaining temporary hit points equal to
twice the level of the spell, or three times the spell's level if the
spell is from your chosen school. Instead of gaining these temporary hit
points yourself, you can use your reaction to grant them to a creature
you can see within 30 feet.
\end{description}

\begin{description}
\item[Energy Retention]
When you expend a spell slot of 2nd-level or higher to cast a wizard
spell from your
\hyperref[wizard-subclass-arcanist-scholarly-speciality]{chosen school},
you regain one expended spell slot. The regained spell slot must be of a
level no more than half the level of the expended spell slot.
\end{description}

\begin{description}
\item[War Magic]
When you cast a wizard spell from your
\hyperref[wizard-subclass-arcanist-scholarly-speciality]{chosen school}
that deals damage, you deal additional damage equal to your Intelligence
modifier (minimum +1) on the first damage roll for that spell.
\end{description}

\subparagraph{Superior
Talent}\label{wizard-subclass-arcanist-superior-talent}

At 10th level, choose one of the following benefits:

\begin{description}
\item[Perfect Control]
When concentrating on a wizard spell of your
\hyperref[wizard-subclass-arcanist-scholarly-speciality]{chosen school},
you only need to roll to maintain concentration when you take damage
from an attack, effect, or spell equal to or greater than your
Intelligence score + your wizard level.
\end{description}

\begin{description}
\item[Secondary Learning]
Choose a second school as your
\hyperref[wizard-subclass-arcanist-scholarly-speciality]{Scholarly
Specialty} and an additional feature from either
\hyperref[wizard-subclass-arcanist-esoteric-talent]{Esoteric Talent} or
\hyperref[wizard-subclass-arcanist-refined-learning]{Refined Learning}.
\end{description}

\begin{description}
\item[Splinter Spell]
Once per rest, when you cast a wizard spell from your
\hyperref[wizard-subclass-arcanist-scholarly-speciality]{chosen school}
that only affects one creature, you can choose to affect an additional
creature within range.
\end{description}

\subparagraph{Specialized
Mastery}\label{wizard-subclass-arcanist-specialized-mastery}

At 14th level, choose one of the following benefits:

\begin{description}
\item[Battle Hardiness]
When concentrating on a wizard spell from your
\hyperref[wizard-subclass-arcanist-scholarly-speciality]{chosen school},
you reduce bludgeoning, piercing, and slashing damage you take by an
amount equal to the level of the spell.
\end{description}

\begin{description}
\item[Heightened Potency]
When you cast a wizard spell from your
\hyperref[wizard-subclass-arcanist-scholarly-speciality]{chosen school},
it is always treated as though it were cast with a spell slot one level
higher (maximum 9th-level) than the one you used, so long as you
expended a spell slot to cast it. The slot you use to cast the spell
must still be at least equal to the level of the spell.
\end{description}

\begin{description}
\item[Precise Understanding]
When you see a creature cast a spell from your
\hyperref[wizard-subclass-arcanist-scholarly-speciality]{chosen school},
you automatically know what spell it's casting. In addition, you have
advantage on saving throws against spells.
\end{description}

\section{Ancestries}\label{_ancestries}

\subsection{Ancestral Traits}\label{_ancestral_traits}

The description of each ancestry includes inherited traits that are
common to members of that ancestry. The following entries appear among
the traits of most ancestries. Some ancestries have variants with traits
of the parent ancestry and variant-specific traits.

\begin{description}
\item[Age]
The age entry notes when an ancestry member becomes an adult and its
expected lifespan. This helps decide your character's age at the game's
start. You can choose any age, which may explain ability scores. For
instance, a young or old character might have low Strength or
Constitution, while advanced age could explain high Intelligence or
Wisdom.
\item[Size]
Characters of most ancestries are Medium, between 4 and 8 feet tall. A
few ancestries are Small (2 to 4 feet tall), and some game rules may
affect them differently. Small characters may struggle with heavy
weapons, as explained in ``Equipment.''
\item[Speed]
Your speed determines how far you can move when traveling
(``Adventuring'') and fighting (``Combat'').
\item[Languages]
By virtue of your ancestry, your character can speak, read, and write
certain languages.
\end{description}

\subsection{Dwarf}\label{_dwarf}

Sturdy and resilient, dwarves are known for their compact build with
strong frames and broad features. Their intricate beards and braids
often represent cultural pride. Dwarves have a reputation for
craftsmanship and have a deep connection to the earth, often favoring
mountainous or underground regions.

\subsubsection{Dwarf Traits}\label{_dwarf_traits}

Your dwarf character has an assortment of inborn abilities, part and
parcel of dwarven nature.

\begin{description}
\item[Age.]
Dwarves mature at the same rate as humans, but they're considered young
until they reach the age of 50. On average, they live about 350 years.
\item[Size.]
Dwarves stand between 4 and 5 feet tall and average about 150 pounds.
Your size is Medium.
\item[Speed.]
Your base walking speed is 25 feet. Your speed is not reduced by wearing
heavy armor.
\item[Darkvision.]
Accustomed to life underground, you have superior vision in dark and dim
conditions. You can see in dim light within 60 feet of you as if it were
bright light, and in darkness as if it were dim light. You can't discern
color in darkness, only shades of gray.
\item[Dwarven Resilience.]
You have advantage on saving throws against poison, and you have
resistance against poison damage.
\end{description}

\subsection{Elf}\label{_elf}

Elves have sharp, angular features and pointed ears that vary widely in
color, often reflecting their connection to nature or magic. They are
long-lived, known for valuing art, knowledge, and harmony with their
environment.

\subsubsection{Elf Traits}\label{_elf_traits}

Your elf character has a variety of natural abilities, the result of
thousands of years of elven refinement.

\begin{description}
\item[Age.]
Elves reach physical maturity around the same age as humans, but
adulthood encompasses worldly experience. They typically claim adulthood
and an adult name around 100 and can live up to 750 years.
\item[Size.]
Elves range from under 5 to over 6 feet tall and tend to have slender
builds. Your size is Medium.
\item[Speed.]
Your base walking speed is 30 feet.
\item[Darkvision.]
Accustomed to twilit forests and the night sky, you have superior vision
in dark and dim conditions. You can see in dim light within 60 feet of
you as if it were bright light, and in darkness as if it were dim light.
You can't discern color in darkness, only shades of gray.
\item[Keen Senses.]
You have proficiency in the Perception skill.
\item[Fey Ancestry.]
You have advantage on saving throws against being charmed, and magic
can't put you to sleep.
\item[Trance.]
Elves meditate and dream deeply for 4 hours daily, remaining
semiconscious, which is called ``trance.'' These dreams are mental
exercises that have become reflexive through practice. After resting
this way, you gain the same benefit as a human from 8 hours of sleep.
\end{description}

\section{Cultures}\label{_cultures}

Choose a culture for your character. While some cultures are closely
associated with a specific ancestry, depending on your character's past,
you may choose any culture for them. Each culture has unique traits. You
gain all traits associated with your chosen culture, unless otherwise
stated. Characters raised in a culture share common traits.

\subsection{Cosmopolitan}\label{_cosmopolitan}

Urban dwellers from this culture value adaptability, social connections,
and quick thinking. They thrive in diverse environments, seamlessly
navigating social circles and leveraging their resourcefulness.

\begin{description}
\item[Discreetly Armed.]
You gain expertise on checks made to persuade others to let you remain
armed or to conceal weapons or items about your person.
\item[Fashion Sense.]
After you spend at least 1 minute observing a creature within 60 feet,
you can use an action to make either an Insight or History check against
a DC equal to the creature's passive Deception check score. On a
success, you learn the following information about that creature:

\begin{itemize}
\item
  Whether the creature has a lower Charisma score than yourself.
\item
  The creature's culture and national origin (if any).
\item
  The creature's social standing in the local majority culture.
\end{itemize}
\item[Skill Versatility.]
You gain proficiency in Persuasion and one other skill of your choice.
\item[Urban Denizen.]
You can make an Investigation check to learn a person's location (or
gain a helpful clue) by discreetly asking in the right places. The
difficulty is DC 15 if they're not hiding, or DC 20 if they're trying to
conceal it.
\item[Well-Connected.]
You gain an extra connection, selected from a background of your choice.
This person is of a different heritage or national origin than yourself.
\item[Languages.]
You can speak, read, write, and sign in Common and two additional
languages.
\end{description}

\subsection{Lone Wanderer}\label{_lone_wanderer}

This culture, characterized by its independent spirit, values
self-reliance and adaptability. Its members are resourceful and embrace
unique paths in life.

\begin{description}
\item[Culture of My Own.]
You gain four skill or tool proficiencies of your choice.
\item[Heirloom.]
Choose one weapon worth 100 gold or less. You begin play with a
masterwork version of that weapon.
\item[Languages.]
You can speak, read, write, and sign Common and two additional
languages.
\end{description}

\section{Level Advancement}\label{_level_advancement}

As your character gains experience points and levels up, they gain
additional features and proficiency bonuses. Each level also grants an
extra Hit Die, which can be rolled and added to your hit point maximum,
or used as a fixed value. When your Constitution modifier increases,
your hit point maximum increases by 1 for each level. The
\hyperref[level-advancement-character-advancement-table]{Character
Advancement table} summarizes the XP needed to level up from 1 to 20 and
the proficiency bonus for each level. Refer to your character's class
description for other level-based improvements.

\begin{longtable}[]{@{}
  >{\raggedright\arraybackslash}p{(\linewidth - 4\tabcolsep) * \real{0.4000}}
  >{\raggedright\arraybackslash}p{(\linewidth - 4\tabcolsep) * \real{0.2000}}
  >{\raggedright\arraybackslash}p{(\linewidth - 4\tabcolsep) * \real{0.4000}}@{}}
\caption{Character Advancement
(table)}\label{level-advancement-character-advancement-table}\tabularnewline
\toprule\noalign{}
\begin{minipage}[b]{\linewidth}\raggedright
Experience Points
\end{minipage} & \begin{minipage}[b]{\linewidth}\raggedright
Level
\end{minipage} & \begin{minipage}[b]{\linewidth}\raggedright
Proficiency Bonus
\end{minipage} \\
\midrule\noalign{}
\endfirsthead
\toprule\noalign{}
\begin{minipage}[b]{\linewidth}\raggedright
Experience Points
\end{minipage} & \begin{minipage}[b]{\linewidth}\raggedright
Level
\end{minipage} & \begin{minipage}[b]{\linewidth}\raggedright
Proficiency Bonus
\end{minipage} \\
\midrule\noalign{}
\endhead
\bottomrule\noalign{}
\endlastfoot
0 & 1 & +2 \\
300 & 2 & +2 \\
900 & 3 & +2 \\
2,700 & 4 & +2 \\
6,500 & 5 & +3 \\
14,000 & 6 & +3 \\
23,000 & 7 & +3 \\
34,000 & 8 & +3 \\
48,000 & 9 & +4 \\
64,000 & 10 & +4 \\
85,000 & 11 & +4 \\
100,000 & 12 & +4 \\
120,000 & 13 & +5 \\
140,000 & 14 & +5 \\
165,000 & 15 & +5 \\
195,000 & 16 & +5 \\
225,000 & 17 & +6 \\
265,000 & 18 & +6 \\
305,000 & 19 & +6 \\
355,000 & 20 & +6 \\
\end{longtable}

\subsection{Multiclassing}\label{_multiclassing}

Multiclassing lets you gain levels in multiple classes, mixing their
abilities to create unique character concepts. You can gain a level in a
new class whenever you advance, instead of your current class. Levels in
all classes add up to determine your character level. For instance,
three Wizard levels and two Fighter levels make you a 5th-level
character.

As you level up, you may stay in your original class with a few levels
in another, or change course entirely. You might even start progressing
in a third or fourth class. Compared to a single-class character of the
same level, you sacrifice focus for versatility.

\subsubsection{Prerequisites}\label{_prerequisites}

To qualify for a new class, meet the ability score prerequisites for
both your current and new classes, as shown in the
\hyperref[multiclassing-prerequisites-table]{Multiclassing Prerequisites
table}. For instance, a Dreadnought multiclassing into Primal must have
13 or higher Strength and Wisdom scores. Without the training of a
beginning character, you must be a quick learner with natural aptitude
reflected by higher-than-average ability scores.

\begin{longtable}[]{@{}
  >{\raggedright\arraybackslash}p{(\linewidth - 2\tabcolsep) * \real{0.2000}}
  >{\raggedright\arraybackslash}p{(\linewidth - 2\tabcolsep) * \real{0.8000}}@{}}
\caption{Multiclassing Prerequisites
(table)}\label{multiclassing-prerequisites-table}\tabularnewline
\toprule\noalign{}
\begin{minipage}[b]{\linewidth}\raggedright
Class
\end{minipage} & \begin{minipage}[b]{\linewidth}\raggedright
Ability Score Minimum
\end{minipage} \\
\midrule\noalign{}
\endfirsthead
\toprule\noalign{}
\begin{minipage}[b]{\linewidth}\raggedright
Class
\end{minipage} & \begin{minipage}[b]{\linewidth}\raggedright
Ability Score Minimum
\end{minipage} \\
\midrule\noalign{}
\endhead
\bottomrule\noalign{}
\endlastfoot
Dreadnought & Strength 13 \\
Bard & Charisma 13 \\
Cleric & Wisdom 13 \\
Primal & Wisdom 13 \\
Fighter & Strength 13 or Dexterity 13 \\
Adept & Dexterity 13 and Wisdom 13 \\
Vanguard & Strength 13 and Charisma 13 \\
Ranger & Dexterity 13 and Wisdom 13 \\
Rogue & Dexterity 13 \\
Sorcerer & Charisma 13 \\
Warlock & Charisma 13 \\
Wizard & Intelligence 13 \\
\end{longtable}

\subsection{Proficiencies}\label{_proficiencies_4}

When you gain your first level in a class other than your initial class,
you gain only some of new class's starting proficiencies, as shown in
the \hyperref[multiclassing-proficiencies-table]{Multiclassing
Proficiencies table}.

\begin{longtable}[]{@{}
  >{\raggedright\arraybackslash}p{(\linewidth - 2\tabcolsep) * \real{0.2000}}
  >{\raggedright\arraybackslash}p{(\linewidth - 2\tabcolsep) * \real{0.8000}}@{}}
\caption{Multiclassing Proficiencies
(table)}\label{multiclassing-proficiencies-table}\tabularnewline
\toprule\noalign{}
\begin{minipage}[b]{\linewidth}\raggedright
Class
\end{minipage} & \begin{minipage}[b]{\linewidth}\raggedright
Proficiencies Gained
\end{minipage} \\
\midrule\noalign{}
\endfirsthead
\toprule\noalign{}
\begin{minipage}[b]{\linewidth}\raggedright
Class
\end{minipage} & \begin{minipage}[b]{\linewidth}\raggedright
Proficiencies Gained
\end{minipage} \\
\midrule\noalign{}
\endhead
\bottomrule\noalign{}
\endlastfoot
Dreadnought & Shields, simple weapons, martial weapons \\
Bard & Light armor, one skill of your choice, one musical instrument of
your choice \\
Cleric & Light armor, medium armor, shields \\
Primal & Light armor, medium armor, shields (Primals will not wear armor
or use shields made of metal) \\
Fighter & Light armor, medium armor, shields, simple weapons, martial
weapons \\
Adept & Simple weapons, shortswords \\
Vanguard & Light armor, medium armor, shields, simple weapons, martial
weapons \\
Ranger & Light armor, medium armor, shields, simple weapons, martial
weapons, one skill from the class's skill list \\
Rogue & Light armor, one skill from the class's skill list, thieves'
tools \\
Sorcerer & --- \\
Warlock & Light armor, simple weapons \\
Wizard & --- \\
\end{longtable}

\subsubsection{Languages}\label{languages}

Your culture determines your default languages that you can read, speak,
write, and sign, provided there is no disability or condition that
prevents you from doing so. Your background may grant access to
additional languages of your choice. Note these languages on your
character sheet.

Choose from the \hyperref[languages-standard-table]{Standard Languages
table}, or a common language in your campaign. With your GM's
permission, you can choose an \hyperref[languages-exotic-table]{exotic
language} if it fits your background.

Some languages are families with many dialects. For instance, the
Primordial language includes Auran, Aquan, Ignan, and Terran dialects
for each elemental plane. Creatures speaking different dialects of the
same language can communicate.

\begin{longtable}[]{@{}
  >{\raggedright\arraybackslash}p{(\linewidth - 4\tabcolsep) * \real{0.3333}}
  >{\raggedright\arraybackslash}p{(\linewidth - 4\tabcolsep) * \real{0.3333}}
  >{\raggedright\arraybackslash}p{(\linewidth - 4\tabcolsep) * \real{0.3333}}@{}}
\caption{Standard Languages
(table)}\label{languages-standard-table}\tabularnewline
\toprule\noalign{}
\begin{minipage}[b]{\linewidth}\raggedright
Language
\end{minipage} & \begin{minipage}[b]{\linewidth}\raggedright
Typical Speakers
\end{minipage} & \begin{minipage}[b]{\linewidth}\raggedright
Script
\end{minipage} \\
\midrule\noalign{}
\endfirsthead
\toprule\noalign{}
\begin{minipage}[b]{\linewidth}\raggedright
Language
\end{minipage} & \begin{minipage}[b]{\linewidth}\raggedright
Typical Speakers
\end{minipage} & \begin{minipage}[b]{\linewidth}\raggedright
Script
\end{minipage} \\
\midrule\noalign{}
\endhead
\bottomrule\noalign{}
\endlastfoot
Common & Humans & Common \\
Dwarvish & Dwarves & Dwarvish \\
Elvish & Elves & Elvish \\
Giant & Ogres, giants & Dwarvish \\
Gnomish & Gnomes & Dwarvish \\
Goblin & Goblinoids & Dwarvish \\
Halfling & Halflings & Common \\
Orc & Orcs & Dwarvish \\
\end{longtable}

\begin{longtable}[]{@{}
  >{\raggedright\arraybackslash}p{(\linewidth - 4\tabcolsep) * \real{0.3333}}
  >{\raggedright\arraybackslash}p{(\linewidth - 4\tabcolsep) * \real{0.3333}}
  >{\raggedright\arraybackslash}p{(\linewidth - 4\tabcolsep) * \real{0.3333}}@{}}
\caption{Exotic Languages
(table)}\label{languages-exotic-table}\tabularnewline
\toprule\noalign{}
\begin{minipage}[b]{\linewidth}\raggedright
Language
\end{minipage} & \begin{minipage}[b]{\linewidth}\raggedright
Typical Speakers
\end{minipage} & \begin{minipage}[b]{\linewidth}\raggedright
Script
\end{minipage} \\
\midrule\noalign{}
\endfirsthead
\toprule\noalign{}
\begin{minipage}[b]{\linewidth}\raggedright
Language
\end{minipage} & \begin{minipage}[b]{\linewidth}\raggedright
Typical Speakers
\end{minipage} & \begin{minipage}[b]{\linewidth}\raggedright
Script
\end{minipage} \\
\midrule\noalign{}
\endhead
\bottomrule\noalign{}
\endlastfoot
Abyssal & Demons & Infernal \\
Celestial & Celestials & Celestial \\
Draconic & Dragons, dragonborn & Draconic \\
Deep Speech & Aboleths, cloakers & --- \\
Infernal & Devils & Infernal \\
Primordial & Elementals & Dwarvish \\
Sylvan & Fey creatures & Elvish \\
Undercommon & Underworld traders & Elvish \\
\end{longtable}

Signing

You must have at least one hand free to communicate by sign, and the
creature you are communicating with must be able to see you. When
attempting to make subtle signs, to remain unnoticed you must succeed on
a Sleight of Hand check against the passive Perception scores of
observers.

\section{Equipment}\label{_equipment_4}

Common coins are made of gold, silver, and copper, with different
denominations based on their metal worth.

A gold piece can buy a bedroll, rope, or a goat. Skilled artisans earn
one gold piece daily. Gold is the standard unit of wealth, though coins
aren't commonly used. When discussing deals worth hundreds or thousands
of gold pieces, transactions usually involve gold bars, letters of
credit, or valuable goods.

A gold piece is worth ten silver pieces, which buy a laborer's work,
lamp oil, or a poor inn's night's rest. A silver piece is worth ten
copper pieces, which buy candles, torches, or chalk.

Unusual coins made of other precious metals, like electrum and platinum,
sometimes appear in treasure hoards. Electrum is worth five silver
pieces, and platinum is worth ten gold pieces.

A standard coin weighs about a third of an ounce, so fifty coins weigh a
pound.

\begin{longtable}[]{@{}
  >{\raggedright\arraybackslash}p{(\linewidth - 10\tabcolsep) * \real{0.2857}}
  >{\raggedright\arraybackslash}p{(\linewidth - 10\tabcolsep) * \real{0.1429}}
  >{\raggedright\arraybackslash}p{(\linewidth - 10\tabcolsep) * \real{0.1429}}
  >{\raggedright\arraybackslash}p{(\linewidth - 10\tabcolsep) * \real{0.1429}}
  >{\raggedright\arraybackslash}p{(\linewidth - 10\tabcolsep) * \real{0.1429}}
  >{\raggedright\arraybackslash}p{(\linewidth - 10\tabcolsep) * \real{0.1429}}@{}}
\caption{Standard Exchange Rates (table)}\tabularnewline
\toprule\noalign{}
\begin{minipage}[b]{\linewidth}\raggedright
Coin
\end{minipage} & \begin{minipage}[b]{\linewidth}\raggedright
CP
\end{minipage} & \begin{minipage}[b]{\linewidth}\raggedright
SP
\end{minipage} & \begin{minipage}[b]{\linewidth}\raggedright
EP
\end{minipage} & \begin{minipage}[b]{\linewidth}\raggedright
GP
\end{minipage} & \begin{minipage}[b]{\linewidth}\raggedright
PP
\end{minipage} \\
\midrule\noalign{}
\endfirsthead
\toprule\noalign{}
\begin{minipage}[b]{\linewidth}\raggedright
Coin
\end{minipage} & \begin{minipage}[b]{\linewidth}\raggedright
CP
\end{minipage} & \begin{minipage}[b]{\linewidth}\raggedright
SP
\end{minipage} & \begin{minipage}[b]{\linewidth}\raggedright
EP
\end{minipage} & \begin{minipage}[b]{\linewidth}\raggedright
GP
\end{minipage} & \begin{minipage}[b]{\linewidth}\raggedright
PP
\end{minipage} \\
\midrule\noalign{}
\endhead
\bottomrule\noalign{}
\endlastfoot
Copper (cp) & \(1\) & \(\frac{1}{10}\) & \(\frac{1}{50}\) &
\(\frac{1}{100}\) & \(\frac{1}{1000}\) \\
Silver (sp) & \(10\) & \(1\) & \(\frac{1}{5}\) & \(\frac{1}{10}\) &
\(\frac{1}{100}\) \\
Electrum (ep) & \(50\) & \(5\) & \(1\) & \(\frac{1}{2}\) &
\(\frac{1}{20}\) \\
Gold (gp) & \(100\) & \(10\) & \(2\) & \(1\) & \(\frac{1}{10}\) \\
Platinum (pp) & \(1000\) & \(100\) & \(20\) & \(10\) & \(1\) \\
\end{longtable}

\subsection{Selling Treasure}\label{_selling_treasure}

Opportunities arise to find treasure, equipment, weapons, armor, and
more in dungeons. You can sell your loot in towns or settlements if you
find buyers.

Undamaged weapons, armor, and other equipment fetch half their cost in
markets. Monsters' weapons and armor are rarely in good condition for
sale.

Selling magic items is tricky. Potions and scrolls are easy to find
buyers for, but other items are rare and expensive, mostly for wealthy
nobles. Magic items are far more valuable than gold and should be
treated as such.

Gems, jewelry, and art objects retain their full value and can be traded
for coin or used as currency. For exceptionally valuable treasures, the
GM may require you to find a buyer in a large town or community.

Trade goods such as grain, salt, and domesticated beasts are sought
after everywhere and so are unlikely to diminish much in value from
place to place.

\subsection{Armor}\label{_armor}

Fantasy gaming worlds have diverse cultures with varying technology
levels, offering adventurers a wide range of armor types, from leather
to chain mail and costly plate armor. The Armor table categorizes these
common armor types into light, medium, and heavy armor, along with their
cost, weight, and other properties. Many warriors also use shields.

\begin{description}
\item[Armor Proficiency.]
Anyone can put on a suit of armor or strap a shield to an arm. Only
those proficient in the armor's use know how to wear it effectively,
however. Your class gives you proficiency with certain types of armor.
If you wear armor that you lack proficiency with, you have disadvantage
on any ability check, saving throw, or attack roll that involves
Strength or Dexterity, and you can't cast spells.
\item[Armor Class (AC).]
Armor protects its wearer from attacks. The armor (and shield) you wear
determines your base Armor Class.
\item[Heavy Armor.]
Heavier armor interferes with the wearer's ability to move quickly,
stealthily, and freely. If the Armor table shows ``Str 13'' or ``Str
15'' in the Strength column for an armor type, the armor reduces the
wearer's speed by 10 feet unless the wearer has a Strength score equal
to or higher than the listed score.
\item[Stealth.]
If the Armor table shows ``Disadvantage'' in the Stealth column, the
wearer has disadvantage on Dexterity (Stealth) checks.
\item[Shields.]
A shield is made from wood or metal and is carried in one hand. Wielding
a shield increases your Armor Class by 2. You can benefit from only one
shield at a time.
\end{description}

\subsubsection{Light Armor}\label{_light_armor}

Made from supple and thin materials, light armor favors agile
adventurers since it offers some protection without sacrificing
mobility. If you wear light armor, you add your Dexterity modifier to
the base number from your armor type to determine your Armor Class.

\begin{description}
\item[Padded.]
Padded armor consists of quilted layers of cloth and batting.
\item[Leather.]
The breastplate and shoulder protectors of this armor are made of
leather that has been stiffened by being boiled in oil. The rest of the
armor is made of softer and more flexible materials.
\item[Studded Leather.]
Made from tough but flexible leather, studded leather is reinforced with
close-set rivets or spikes.
\end{description}

\subsubsection{Medium Armor}\label{_medium_armor}

Medium armor offers more protection than light armor, but it also
impairs movement more. If you wear medium armor, you add your Dexterity
modifier, to a maximum of +2, to the base number from your armor type to
determine your Armor Class.

\begin{description}
\item[Hide.]
This crude armor consists of thick furs and pelts.
\item[Chain Shirt.]
Made of interlocking metal rings, a chain shirt is worn between layers
of clothing or leather. This armor offers modest protection to the
wearer's upper body and allows the sound of the rings rubbing against
one another to be muffled by outer layers.
\item[Scale Mail.]
This armor consists of a coat and leggings (and perhaps a separate
skirt) of leather covered with overlapping pieces of metal, much like
the scales of a fish. The suit includes gauntlets.
\item[Breastplate.]
This armor consists of a fitted metal chest piece worn with supple
leather. Although it leaves the legs and arms relatively unprotected,
this armor provides good protection for the wearer's vital organs while
leaving the wearer relatively unencumbered.
\item[Half Plate.]
Half plate consists of shaped metal plates that cover most of the
wearer's body. It does not include leg protection beyond simple greaves
that are attached with leather straps.
\end{description}

\subsubsection{Heavy Armor}\label{_heavy_armor}

Of all the armor categories, heavy armor offers the best protection.
These suits of armor cover the entire body and are designed to stop a
wide range of attacks. Only proficient warriors can manage their weight
and bulk.

Heavy armor doesn't let you add your Dexterity modifier to your Armor
Class, but it also doesn't penalize you if your Dexterity modifier is
negative.

\begin{description}
\item[Ring Mail.]
This armor is leather armor with heavy rings sewn into it. The rings
help reinforce the armor against blows from swords and axes. Ring mail
is inferior to chain mail, and it's usually worn only by those who can't
afford better armor.
\item[Chain Mail.]
Made of interlocking metal rings, chain mail includes a layer of quilted
fabric worn underneath the mail to prevent chafing and to cushion the
impact of blows. The suit includes gauntlets.
\item[Splint.]
This armor is made of narrow vertical strips of metal riveted to a
backing of leather that is worn over cloth padding. Flexible chain mail
protects the joints.
\item[Plate.]
Plate consists of shaped, interlocking metal plates to cover the entire
body. A suit of plate includes gauntlets, heavy leather boots, a visored
helmet, and thick layers of padding underneath the armor. Buckles and
straps distribute the weight over the body.
\end{description}

\begin{longtable}[]{@{}
  >{\raggedright\arraybackslash}p{(\linewidth - 10\tabcolsep) * \real{0.2222}}
  >{\raggedright\arraybackslash}p{(\linewidth - 10\tabcolsep) * \real{0.1111}}
  >{\raggedright\arraybackslash}p{(\linewidth - 10\tabcolsep) * \real{0.2222}}
  >{\raggedright\arraybackslash}p{(\linewidth - 10\tabcolsep) * \real{0.1111}}
  >{\raggedright\arraybackslash}p{(\linewidth - 10\tabcolsep) * \real{0.2222}}
  >{\raggedright\arraybackslash}p{(\linewidth - 10\tabcolsep) * \real{0.1111}}@{}}
\caption{Armor (table)}\tabularnewline
\toprule\noalign{}
\begin{minipage}[b]{\linewidth}\raggedright
Armor
\end{minipage} & \begin{minipage}[b]{\linewidth}\raggedright
Cost
\end{minipage} & \begin{minipage}[b]{\linewidth}\raggedright
Armor Class (AC)
\end{minipage} & \begin{minipage}[b]{\linewidth}\raggedright
Strength
\end{minipage} & \begin{minipage}[b]{\linewidth}\raggedright
Stealth
\end{minipage} & \begin{minipage}[b]{\linewidth}\raggedright
Weight
\end{minipage} \\
\midrule\noalign{}
\endfirsthead
\toprule\noalign{}
\begin{minipage}[b]{\linewidth}\raggedright
Armor
\end{minipage} & \begin{minipage}[b]{\linewidth}\raggedright
Cost
\end{minipage} & \begin{minipage}[b]{\linewidth}\raggedright
Armor Class (AC)
\end{minipage} & \begin{minipage}[b]{\linewidth}\raggedright
Strength
\end{minipage} & \begin{minipage}[b]{\linewidth}\raggedright
Stealth
\end{minipage} & \begin{minipage}[b]{\linewidth}\raggedright
Weight
\end{minipage} \\
\midrule\noalign{}
\endhead
\bottomrule\noalign{}
\endlastfoot
\multicolumn{6}{@{}>{\centering\arraybackslash}p{(\linewidth - 10\tabcolsep) * \real{1.0000} + 10\tabcolsep}@{}}{%
\textbf{Light Armor}} \\
Padded & 5 gp & 11 + Dex modifier & --- & Disadvantage & 8 lb. \\
Leather & 10 gp & 11 + Dex modifier & --- & --- & 10 lb. \\
Studded leather & 45 gp & 12 + Dex modifier & --- & --- & 13 lb. \\
\multicolumn{6}{@{}>{\centering\arraybackslash}p{(\linewidth - 10\tabcolsep) * \real{1.0000} + 10\tabcolsep}@{}}{%
\textbf{Medium Armor}} \\
Hide & 10 gp & 12 + Dex modifier (max 2) & --- & --- & 12 lb. \\
Chain shirt & 50 gp & 13 + Dex modifier (max 2) & --- & --- & 20 lb. \\
Scale mail & 50 gp & 14 + Dex modifier (max 2) & --- & Disadvantage & 45
lb. \\
Breastplate & 400 gp & 14 + Dex modifier (max 2) & --- & --- & 20 lb. \\
Half plate & 750 gp & 15 + Dex modifier (max 2) & --- & Disadvantage &
40 lb. \\
\multicolumn{6}{@{}>{\centering\arraybackslash}p{(\linewidth - 10\tabcolsep) * \real{1.0000} + 10\tabcolsep}@{}}{%
\textbf{Heavy Armor}} \\
Ring mail & 30 gp & 14 & --- & Disadvantage & 40 lb. \\
Chain mail & 75 gp & 16 & Str 13 & Disadvantage & 55 lb. \\
Splint & 200 gp & 17 & Str 15 & Disadvantage & 60 lb. \\
Plate & 1,500 gp & 18 & Str 15 & Disadvantage & 65 lb. \\
\multicolumn{6}{@{}>{\centering\arraybackslash}p{(\linewidth - 10\tabcolsep) * \real{1.0000} + 10\tabcolsep}@{}}{%
\textbf{Shield}} \\
Shield & 10 gp & +2 & --- & --- & 6 lb. \\
\end{longtable}

\subsubsection{Getting Into and Out of
Armor}\label{_getting_into_and_out_of_armor}

The time it takes to don or doff armor depends on the armor's category.

\begin{description}
\item[Don.]
This is the time it takes to put on armor. You benefit from the armor's
AC only if you take the full time to don the suit of armor.
\item[Doff.]
This is the time it takes to take off armor. If you have help, reduce
this time by half.
\end{description}

\begin{longtable}[]{@{}
  >{\raggedright\arraybackslash}p{(\linewidth - 4\tabcolsep) * \real{0.3333}}
  >{\raggedright\arraybackslash}p{(\linewidth - 4\tabcolsep) * \real{0.3333}}
  >{\raggedright\arraybackslash}p{(\linewidth - 4\tabcolsep) * \real{0.3333}}@{}}
\caption{Donning and Doffing Armor (table)}\tabularnewline
\toprule\noalign{}
\begin{minipage}[b]{\linewidth}\raggedright
Category
\end{minipage} & \begin{minipage}[b]{\linewidth}\raggedright
Don
\end{minipage} & \begin{minipage}[b]{\linewidth}\raggedright
Doff
\end{minipage} \\
\midrule\noalign{}
\endfirsthead
\toprule\noalign{}
\begin{minipage}[b]{\linewidth}\raggedright
Category
\end{minipage} & \begin{minipage}[b]{\linewidth}\raggedright
Don
\end{minipage} & \begin{minipage}[b]{\linewidth}\raggedright
Doff
\end{minipage} \\
\midrule\noalign{}
\endhead
\bottomrule\noalign{}
\endlastfoot
Light Armor & 1 minute & 1 minute \\
Medium Armor & 5 minutes & 1 minute \\
Heavy Armor & 10 minutes & 5 minutes \\
Shield & 1 action & 1 action \\
\end{longtable}

\section{Spellcasting}\label{_spellcasting}

Magic rules fantasy gaming worlds, often as spells. This section covers
spellcasting. Character classes and monsters have unique spell learning
and preparation methods. Regardless of origin, spells follow these
rules.

\subsection{What Is a Spell?}\label{_what_is_a_spell}

A spell is a discrete magical effect, a single shaping of magical
energies in the multiverse. Casting a spell involves plucking, pinning,
vibrating, and releasing invisible strands of magic to unleash the
desired effect, usually in seconds. Spells can be versatile tools,
weapons, or protective wards, dealing damage, undoing it, imposing or
removing conditions, draining life energy, or restoring it. Thousands of
spells have been created throughout history, many forgotten. Some might
be recorded in ancient spellbooks or trapped in the minds of dead gods,
while others could be reinvented by powerful characters.

\subsection{Concentration}\label{_concentration}

Some spells require concentration to maintain their magic. If you lose
concentration, the spell ends. If a spell needs concentration, its
Duration entry specifies how long you can concentrate. You can end
concentration anytime. Normal activity doesn't interfere. The following
can break concentration:

\begin{description}
\item[Casting another spell that requires concentration.]
You lose concentration on a spell if you cast another spell that
requires concentration. You can't concentrate on two spells at once.
\item[Taking damage.]
Whenever you take damage while concentrating on a spell, make a
Constitution saving throw to maintain concentration. The DC is 10 or
half the damage, whichever is higher. If you take damage from multiple
sources, make a separate saving throw for each.
\item[Being incapacitated or killed.]
You lose concentration on a spell if incapacitated or die. The GM may
also require a DC 10 Constitution saving throw to maintain concentration
on a spell due to certain environmental phenomena, like a crashing wave
on a storm-tossed ship.
\end{description}

\subsection{Spell Lists}\label{_spell_lists}

\subsubsection{Vanguard Spells}\label{_vanguard_spells}

\paragraph{1st Level}\label{_1st_level}

\begin{itemize}
\item
  Bless
\item
  Command
\item
  Cure Wounds
\item
  Detect Evil and Good
\item
  Detect Magic
\item
  Detect Poison and Disease
\item
  Divine Favor
\item
  Heroism
\item
  Protection from Evil and Good
\item
  Purify Food and Drink
\item
  Shield of Faith
\end{itemize}

\paragraph{2nd Level}\label{_2nd_level}

\begin{itemize}
\item
  Aid
\item
  Branding Smite
\item
  Find Steed
\item
  Lesser Restoration
\item
  Locate Object
\item
  Magic Weapon
\item
  Protection from Poison
\item
  Zone of Truth
\end{itemize}

\paragraph{3rd Level}\label{_3rd_level}

\begin{itemize}
\item
  Create Food and Water
\item
  Daylight
\item
  Dispel Magic
\item
  Magic Circle
\item
  Remove Curse
\item
  Revivify
\end{itemize}

\paragraph{4th Level}\label{_4th_level}

\begin{itemize}
\item
  Banishment
\item
  Locate Creature
\end{itemize}

\paragraph{5th Level}\label{_5th_level}

\begin{itemize}
\item
  Dispel Evil and Good
\item
  Raise Dead
\end{itemize}

\subsubsection{Wizard Spells}\label{_wizard_spells}

\paragraph{Cantrips (0 Level)}\label{_cantrips_0_level}

\begin{itemize}
\item
  \hyperref[spell-acid-splash]{Acid Splash}
\item
  Chill Touch
\item
  Dancing Lights
\item
  Fire Bolt
\item
  Friends
\item
  Light
\item
  Mage Hand
\item
  Mending
\item
  Message
\item
  Minor Illusion
\item
  Poison Spray
\item
  Prestidigitation
\item
  Ray of Frost
\item
  Shocking Grasp
\item
  True Strike
\end{itemize}

\paragraph{1st Level}\label{_1st_level_2}

\begin{itemize}
\item
  Alarm
\item
  Burning Hands
\item
  Charm Person
\item
  Color Spray
\item
  Comprehend Languages
\item
  Detect Magic
\item
  Disguise Self
\item
  Expeditious Retreat
\item
  False Life
\item
  Feather Fall
\item
  Find Familiar
\item
  Floating Disk
\item
  Fog Cloud
\item
  Hideous Laughter
\item
  Illusory Script
\item
  Iz'zart's Swarm Limb
\item
  Longstrider
\item
  Magic Missile
\item
  Shield
\item
  Thunderwave
\end{itemize}

\paragraph{2nd Level}\label{_2nd_level_2}

\begin{itemize}
\item
  Acid Arrow
\item
  Arcane Lock
\item
  Blindness/Deafness
\item
  Continual Flame
\item
  Darkvision
\item
  Enlarge/Reduce
\item
  Gentle Repose
\item
  Hold Person
\item
  Knock
\item
  Locate Object
\item
  Magic Weapon
\item
  Misty Step
\item
  Rope Trick
\item
  See Invisibility
\item
  Spider Climb
\item
  Web
\end{itemize}

\paragraph{3rd Level}\label{_3rd_level_2}

\begin{itemize}
\item
  \hyperref[spell-animate-dead]{Animate Dead}
\item
  Blink
\item
  Clairvoyance
\item
  Counterspell
\item
  Dispel Magic
\item
  Fear
\item
  Fireball
\item
  Fly
\item
  Gaseous Form
\item
  Glyph of Warding
\item
  Haste
\item
  Hypnotic Pattern
\item
  Lightning Bolt
\item
  Magic Circle
\item
  Major Image
\item
  Nondetection
\item
  Phantom Steed
\item
  Protection from Energy
\item
  Remove Curse
\item
  Sending
\item
  Sleet Storm
\item
  Slow
\item
  Stinking Cloud
\item
  Tiny Hut
\item
  Tongues
\item
  Vampiric Touch
\item
  Water Breathing
\end{itemize}

\paragraph{4th Level}\label{_4th_level_2}

\begin{itemize}
\item
  Arcane Eye
\item
  Banishment
\item
  Black Tentacles
\item
  Blight
\item
  Charm Monster
\item
  Confusion
\item
  Conjure Minor Elementals
\item
  Control Water
\item
  Dimension Door
\item
  Fabricate
\item
  Faithful Hound
\item
  Fire Shield
\item
  Greater Invisibility
\item
  Hallucinatory Terrain
\item
  Ice Storm
\item
  Locate Creature
\item
  Phantasmal Killer
\item
  Polymorph
\item
  Private Sanctum
\item
  Resilient Sphere
\item
  Secret Chest
\item
  Stone Shape
\item
  Stoneskin
\item
  Wall of Fire
\end{itemize}

\paragraph{5th Level}\label{_5th_level_2}

\begin{itemize}
\item
  \hyperref[spell-animate-objects]{Animate Objects}
\item
  Arcane Hand
\item
  Cone of Cold
\item
  Contact Other Plane
\item
  Dominate Person
\item
  Geas
\item
  Legend Lore
\item
  Modify Memory
\item
  Planar Binding
\item
  Scrying
\item
  Seeming
\item
  Telepathic Bond
\item
  Wall of Force
\end{itemize}

\paragraph{6th Level}\label{_6th_level}

\begin{itemize}
\item
  Chain Lightning
\item
  Create Undead
\item
  Eyebite
\item
  Globe of Invulnerability
\item
  Instant Summons
\item
  Magic Jar
\item
  Move Earth
\item
  Sunbeam
\item
  Wall of Ice
\end{itemize}

\paragraph{7th Level}\label{_7th_level}

\begin{itemize}
\item
  Arcane Sword
\item
  Etherealness
\item
  Forcecage
\item
  Mirage Arcane
\item
  Prismatic Spray
\item
  Project Image
\item
  Reverse Gravity
\item
  Sequester
\item
  Simulacrum
\item
  Symbol
\item
  Teleport
\end{itemize}

\paragraph{8th Level}\label{_8th_level}

\begin{itemize}
\item
  Antimagic Field
\item
  Antipathy/Sympathy
\item
  Clone
\item
  Control Weather
\item
  Demiplane
\item
  Dominate Monster
\item
  Feeblemind
\item
  Incendiary Cloud
\item
  Maze
\item
  Mind Blank
\item
  Power Word Stun
\item
  Sunburst
\end{itemize}

\paragraph{9th Level}\label{_9th_level}

\begin{itemize}
\item
  Astral Projection
\item
  Foresight
\item
  Gate
\item
  Imprisonment
\item
  Meteor Swarm
\item
  Power Word Kill
\item
  Prismatic Wall
\item
  Shapechange
\item
  Time Stop
\item
  True Polymorph
\item
  Weird
\item
  Wish
\end{itemize}

\section{Spell Descriptions}\label{_spell_descriptions}

\subsection{Acid Splash}\label{spell-acid-splash}

\emph{Conjuration cantrip}

\begin{longtable}[]{@{}
  >{\raggedright\arraybackslash}p{(\linewidth - 2\tabcolsep) * \real{0.1500}}
  >{\raggedright\arraybackslash}p{(\linewidth - 2\tabcolsep) * \real{0.8500}}@{}}
\toprule\noalign{}
\endhead
\bottomrule\noalign{}
\endlastfoot
Casting Time & 1 action \\
Range & 60 feet \\
Target & Up to two creatures within 5 feet of each other \\
Components & V, S \\
Duration & Instantaneous \\
Saving Throw & Dexterity negates \\
\end{longtable}

A stinking bubble of acid is conjured out of thin air to fly at the
targets, dealing 1d6 acid damage.

This spell's damage increases by 1d6 when you reach 5th level (2d6),
11th level (3d6), and 17th level (4d6).

\subsection{Animate Dead}\label{spell-animate-dead}

\emph{3rd-level necromancy}

\begin{longtable}[]{@{}
  >{\raggedright\arraybackslash}p{(\linewidth - 2\tabcolsep) * \real{0.1500}}
  >{\raggedright\arraybackslash}p{(\linewidth - 2\tabcolsep) * \real{0.8500}}@{}}
\toprule\noalign{}
\endhead
\bottomrule\noalign{}
\endlastfoot
Casting Time & 1 minute \\
Range & Touch \\
Components & V, S, M (two copper coins) \\
Duration & Instantaneous \\
\end{longtable}

You animate a mortal's remains to become your undead servant.

If the spell is cast upon bones you create a skeleton, and if cast upon
a corpse you can choose to create a skeleton or a zombie. The GM has the
undead's statistics.

While it is within 60 feet you can use a bonus action to mentally
command any undead you created with this spell. When you command
multiple undead using this spell, you must give them all the same
command. You may decide the creature's exact action and move, or you can
issue a general command, such as guarding an area, which it follows
until the task is complete or you issue it a new command. If not given a
command, the undead only defends itself.

The undead is under your control for 24 hours unless you cast this spell
on it before the spell ends to maintain control of it for another 24
hours. Casting the spell in this way reasserts control over up to 4 of
your previously-animated undead instead of animating a new one. When no
longer under your control, the undead no longer obeys your commands.

\begin{description}
\item[At Higher Levels.]
You create or maintain control over 2 additional undead for each slot
level above 3rd.
\end{description}

\subsection{Animate Objects}\label{spell-animate-objects}

\emph{5th-level transmutation}

\begin{longtable}[]{@{}
  >{\raggedright\arraybackslash}p{(\linewidth - 2\tabcolsep) * \real{0.1500}}
  >{\raggedright\arraybackslash}p{(\linewidth - 2\tabcolsep) * \real{0.8500}}@{}}
\toprule\noalign{}
\endhead
\bottomrule\noalign{}
\endlastfoot
Casting Time & 1 action \\
Range & Long (120 feet) \\
Components & V, S \\
Duration & Concentration (1 minute) \\
\end{longtable}

Choose up to 6 unattended nonmagical Small or Tiny objects. You may also
choose larger objects; treat Medium objects as 2 objects, Large objects
as 3 objects, and Huge objects as 6 objects.

Until the spell ends or a target is reduced to 0 hit points, you animate
the targets and turn them into constructs under your control.

Each construct has Constitution 10, Intelligence 3, Wisdom 3, and
Charisma 1, as well as a flying speed of 30 feet and the ability to
hover (if securely fastened to something larger, it has a Speed of 0),
and blindsight to a range of 30 feet (blind beyond that distance).
Otherwise a construct's statistics are determined by its size.

If you animate 4 or more Small or Tiny objects, instead of controlling
each construct individually they function as a construct swarm. Add
together all swarm's total hit points. Attacks against a construct swarm
deal half damage. The construct swarm reverts to individual constructs
when it is reduced to 15 hit points or less.

You can use a bonus action to mentally command any construct made with
this spell while within 500 feet of it. When you command multiple
constructs using this spell, you must give them all the same command.
You may decide the creature's exact action and move, or you can issue a
general command, such as guarding an area, which it follows until the
task is complete or you issue it a new command. If not given a command,
the construct only defends itself.

When you command a construct to attack, it makes a slam, a melee attack,
against a creature within 5 feet of it. On a hit the construct deals
bludgeoning, piercing, or slashing damage appropriate to its shape.

When the construct drops to 0 hit points, any excess damage carries over
to its inanimate object form.

\begin{description}
\item[At Higher Levels.]
You can animate 2 additional Small or Tiny objects for each slot level
above 5th.
\end{description}

\begin{longtable}[]{@{}
  >{\raggedright\arraybackslash}p{(\linewidth - 10\tabcolsep) * \real{0.2000}}
  >{\raggedright\arraybackslash}p{(\linewidth - 10\tabcolsep) * \real{0.1000}}
  >{\raggedright\arraybackslash}p{(\linewidth - 10\tabcolsep) * \real{0.1000}}
  >{\raggedright\arraybackslash}p{(\linewidth - 10\tabcolsep) * \real{0.4000}}
  >{\raggedright\arraybackslash}p{(\linewidth - 10\tabcolsep) * \real{0.1000}}
  >{\raggedright\arraybackslash}p{(\linewidth - 10\tabcolsep) * \real{0.1000}}@{}}
\caption{Animate Objects Table}\tabularnewline
\toprule\noalign{}
\begin{minipage}[b]{\linewidth}\raggedright
Size
\end{minipage} & \begin{minipage}[b]{\linewidth}\centering
HP
\end{minipage} & \begin{minipage}[b]{\linewidth}\centering
AC
\end{minipage} & \begin{minipage}[b]{\linewidth}\raggedright
Attack
\end{minipage} & \begin{minipage}[b]{\linewidth}\centering
STR
\end{minipage} & \begin{minipage}[b]{\linewidth}\centering
DEX
\end{minipage} \\
\midrule\noalign{}
\endfirsthead
\toprule\noalign{}
\begin{minipage}[b]{\linewidth}\raggedright
Size
\end{minipage} & \begin{minipage}[b]{\linewidth}\centering
HP
\end{minipage} & \begin{minipage}[b]{\linewidth}\centering
AC
\end{minipage} & \begin{minipage}[b]{\linewidth}\raggedright
Attack
\end{minipage} & \begin{minipage}[b]{\linewidth}\centering
STR
\end{minipage} & \begin{minipage}[b]{\linewidth}\centering
DEX
\end{minipage} \\
\midrule\noalign{}
\endhead
\bottomrule\noalign{}
\endlastfoot
Tiny & 5 & 14 & +6 to hit, 1d4 -- 3 damage & 4 & 18 \\
Small & 10 & 12 & +4 to hit, 1d6 -- 2 damage & 6 & 14 \\
Swarm of Tiny and Small & varies & 13 & +5 to hit, 2d6 damage (1d6
damage if bloodied) & 5 & 16 \\
Medium & 20 & 11 & +3 to hit, 1d8 damage & 10 & 12 \\
Large & 40 & 10 & +4 to hit, 2d8 + 2 damage & 14 & 10 \\
Huge & 80 & 8 & +6 to hit, 2d12 + 4 damage & 18 & 6 \\
\end{longtable}

\section{Credits}\label{_credits}

\begin{longtable}[]{@{}
  >{\raggedright\arraybackslash}p{(\linewidth - 2\tabcolsep) * \real{0.1500}}
  >{\raggedright\arraybackslash}p{(\linewidth - 2\tabcolsep) * \real{0.8500}}@{}}
\toprule\noalign{}
\endhead
\bottomrule\noalign{}
\endlastfoot
Project Lead \& Writer & Dale Critchley
\url{https://wyrmworkspublishing.com} \\
Collaborator, Writer, \& Editor & PJ Coffey
\url{https://homebrewandhacking.com/} \\
Layout \& Accessibility & Chris Hopper
\url{https://bio.link/chrishopper} \\
\end{longtable}

\subsection{Artists}\label{_artists}

\subsection{Primary Sources}\label{_primary_sources}

This project was made possible by Sly Flourish, EN Publishing,
KibblesTasty, and Wizards of the Coast, who generously made content
available under a Creative Commons license. We are deeply grateful for
their contributions.

\section{Legal}\label{_legal}

This work includes material taken from the A5E System Reference Document
(A5ESRD) by EN Publishing and available at A5ESRD.com, based on Level
Up: Advanced 5th Edition, available at www.levelup5e.com{[}{]}. The
A5ESRD is licensed under the Creative Commons Attribution 4.0
International License available at
\url{https://creativecommons.org/licenses/by/4.0/legalcode}.

This work includes material taken from the System Reference Document 5.1
(``SRD 5.1'') by Wizards of the Coast LLC and available at
\url{https://dnd.Wizards.com/resources/systems-reference-document}. The
SRD 5.1 is licensed under the Creative Commons Attribution 4.0
International License available at
\url{https://creativecommons.org/licenses/by/4.0/legalcode}.

This work includes material taken from the Lazy GM's Resource Document
by Michael E. Shea of SlyFlourish.com, available under a Creative
Commons Attribution 4.0 International License.

This work includes content from Kibbles' Compendium of Legends and
Legacies by KibblesTasty Homebrew LLC and available at
\url{https://www.kthomebrew.com/krd}. The Kibbles' Compendium of Legends
and Legacies is licensed under the Creative Commons Attribution 4.0
International License (CC-BY-4.0) available at
\url{https://creativecommons.org/licenses/by/4.0/legalcode}.

The text of Free5e Player's Guide © 2025 by Wyrmworks Publishing and
available at \url{https://free5e.com} is licensed under Creative Commons
Attribution 4.0 International. To view a copy of this license, visit
\url{https://creativecommons.org/licenses/by/4.0/} You are free to use
this content in any manner permitted by that license as long as you
include the following attribution statement in your own work:

This work includes material adapted from the \textbf{Free5e Player's
Guide}, © 2025 by Wyrmworks Publishing, and available at
\url{https://free5e.com}. The Free5e Player's Guide is licensed under
the Creative Commons Attribution 4.0 International License (CC-BY-4.0).
To view a copy of this license, visit
\url{https://creativecommons.org/licenses/by/4.0/}.

This adaptation also includes material originally taken from:

\begin{itemize}
\item
  The \textbf{A5E System Reference Document (A5ESRD)} by EN Publishing,
  available at A5ESRD.com and licensed under CC-BY-4.0.
\item
  The \textbf{System Reference Document 5.1 (SRD 5.1)} by Wizards of the
  Coast LLC, available at
  \url{https://dnd.Wizards.com/resources/systems-reference-document} and
  licensed under CC-BY-4.0.
\item
  The \textbf{Lazy GM's Resource Document} by Michael E. Shea of
  SlyFlourish.com, licensed under CC-BY-4.0.
\item
  The \textbf{Kibbles' Compendium of Legends and Legacies} by
  KibblesTasty Homebrew LLC, available at
  \url{https://www.kthomebrew.com/krd} and licensed under CC-BY-4.0.
\end{itemize}

All artwork contained in this book is licensed under CC-BY-4.0 or CC-0.
To use that artwork outside of this project, see the accompanying
document with specific licensing and credit information. If you received
this copy without that document, you can obtain a copy at
\url{https://free5e.com}.
