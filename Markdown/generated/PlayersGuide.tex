\section{}

\emph{Highlighted Portions} are scheduled for significant
modification.\\
Sign up to receive email updates at
\url{https://wyrmworkspublishing.com/hoard}\\
Follow and Back the Kickstarter to make this a reality at
\url{https://www.kickstarter.com/projects/wyrmworkspublishing/free5e-a-free-open-source-dungeons-and-dragons-alternative?ref=dmx0j9}

\section{Preface}\label{PlayersGuide_preface}

\subsection{Introduction}\label{Introduction_introduction}

Welcome to the Free5e Player's Guide, brought to you by Wyrmworks
Publishing. Our mission is to enhance inclusivity and accessibility in
tabletop roleplaying. This guide is designed to help you create unique
characters, embark on exciting adventures, and explore worlds where
everyone can play, create, and share.

\subsubsection{How to Play}\label{Introduction_how-to-play}

Playing a tabletop roleplaying game (TTRPG) is all about imagining a
character in a fantasy world and using dice to determine what happens.
Here's how it works:

\begin{itemize}
\item
  \textbf{Choose a Game Master (GM).}\\
  One player is the GM, who creates the story, controls the world, and
  describes everything and everyone around you.
\item
  \textbf{Create Your Character.}\\
  Choose an ancestry, class, and background for your character. This is
  your hero in the game world!
\item
  \textbf{Playing an Encounter}\\
  On your turn, you describe what your character does. Do you swing your
  sword, cast a spell, or talk to someone? To see whether you succeed,
  you roll a 20-sided die (d20) and add your ability score modifier and,
  maybe, a proficiency bonus. Higher rolls are better!
\end{itemize}

\textbf{Example of play}

\textbf{\emph{Grey (GM):}} The village elder has asked you to
investigate a strange ruin in the nearby forest. After hours of travel,
you arrive at a crumbling stone structure covered in moss. Inside, the
air is damp, and faint echoes of scratching come from within the stony
depths.. What do you do?

\textbf{\emph{Susan (playing Sylvana, a halfling Bard):}} I step in
quietly and take a closer look at the walls. Do I recognize any symbols
or writing?

\textbf{\emph{GM:}} Make an Investigation check, using Intelligence.

\textbf{\emph{Susan:}} (Rolls d20) That's a 12, plus 1 for Intelligence
and 2 from my Investigation proficiency, so 15 total.

\textbf{\emph{GM:}} You recognize some faded symbols of an ancient order
dedicated to protecting the forest. This was likely a Primal temple.

\textbf{\emph{Owen (playing Osmus, a human Ranger):}} I listen closely
to pinpoint where the scratching is coming from. Perception check?

\textbf{\emph{GM:}} Go ahead.

\textbf{\emph{Owen:}} (Rolls d20) I got a 10, plus 4 for Wisdom is 14.

\textbf{\emph{GM:}} The scratching is coming from behind a door at the
far end of a corridor leading inside the temple.

\textbf{\emph{Sylvana:}} I cautiously open the door and peek inside.

\textbf{\emph{GM:}} The door creaks open, revealing dim light glinting
off something large in the shadows. Its long, slimy tentacles sway as it
shifts, and you hear its beak click. It hisses, sensing you. Perched on
the ceiling, it stares down.

\textbf{\emph{Osmus:}} What is that thing? It looks dangerous!

\textbf{\emph{GM:}} It's certainly not friendly. Roll Initiative! (Rolls
d20) With its Dexterity bonus, it gets a 9.

\textbf{\emph{Osmus:}} (Rolls d20) 9, plus 3 for Dexterity is 12.

\textbf{\emph{Sylvana:}} (Rolls d20) 15! I'm first! I try to confuse it
with a quick spell. I cast Vicious Mockery, shouting, ``You look like
something the forest spit out!'' It needs to make a Wisdom saving throw.

\textbf{\emph{GM:}} (Rolls for the creature) That's a 6. It fails.

\textbf{\emph{Sylvana:}} (Rolls 1d4) It takes 3 psychic damage and has
disadvantage on its next attack!

\textbf{\emph{GM:}} The creature seems momentarily stunned, its many
eyes narrowing as it hisses. Osmus, your turn!

\textbf{\emph{Osmus:}} I fire an arrow at it! (Rolls d20) That's a 17,
plus 5 to hit. Does that hit?

\textbf{\emph{GM:}} Yes, that hits. Roll for damage.

\textbf{\emph{Osmus:}} (Rolls 1d8) That's a 2, but I add my Dexterity so
that's a total 5 damage!

\textbf{\emph{GM:}} The arrow strikes true, but the creature's tough
hide absorbs some of the blow. It lunges with its tentacles!

\textbf{\emph{GM (as the creature):}} (Rolls to attack Osmus with the
tentacles) That's a 22 to hit versus your Armor Class.

\textbf{\emph{Osmus:}} Ouch, I've only got 15 so that hits!

\textbf{\emph{GM:}} (Rolls for damage) You take 10 bludgeoning damage,
and I need you to make a Strength saving throw to avoid being pulled in.

\textbf{\emph{Osmus:}} (Rolls d20) That's a 14, plus 3 for Strength, so
17.

\textbf{\emph{GM:}} You hold your ground, but the creature's tentacles
are still trying to wrap around you. It'll try again. What's your next
move?

\textbf{\emph{Sylvana:}} I step back and cast Command on the creature
and shout ``Flee!'' to force it to flee! It needs to make a Wisdom
saving throw, DC 13.

\textbf{\emph{GM:}} (Rolls for the creature) That's a 9. It fails!

\textbf{\emph{Sylvana:}} It must use its reaction to move as far away as
possible!

\textbf{\emph{GM:}} The creature screeches in pain, skittering across
the ceiling to the far corner, giving you space. It looks weakened but
still dangerous.

\textbf{\emph{Osmus:}} Let's finish this! The battle continues, with the
party using teamwork and clever spells to face down the fearsome
creature!

In every game, describe your actions, roll dice to succeed, and react to
the unfolding story. The rules guide you, but the fun comes from your
shared stories. Let your imagination soar!

\subsection{Creating a
Character}\label{Creating_a_Character_creating-a-character}

As a player, begin by creating a character on the character sheet in the
back of this book or a color-coded one at {[}LINK{]}. If you're using a
paper copy, we recommend writing in an erasable medium like pencil.

\begin{enumerate}
\def\labelenumi{\arabic{enumi}.}
\item
  Think of a fantasy character concept that you'd like to play
\item
  Determine ability scores
\item
  Choose your class
\item
  Choose your ancestry
\item
  Choose your heritage
\item
  Choose your background
\item
  Choose your starting equipment
\item
  Choose starting spells if applicable
\item
  Add details like appearance, personality, etc. Consider drawing a
  picture of your character. It doesn't have to be fancy! It's just for
  your friends! (If you'd like to commission a professional portrait of
  your character, check the credits of this book for some great artists
  who love illustrating characters!)
\end{enumerate}

\subsubsection{Determine Ability
Scores}\label{Creating_a_Character_determine-ability-scores}

Use the following scores: 15, 14, 13, 12, 10, 8. Assign each of these
numbers to one of your character's six abilities: Strength, Dexterity,
Constitution, Intelligence, Wisdom, and Charisma. Then add 3 points to
the character's abilities, no more than two to any single ability.

\subsubsection{Variant: Ability Score Point
Buy}\label{Creating_a_Character_variant-ability-score-point-buy}

You have 27 points to spend on ability scores. The cost of each score is
shown on the Ability Score Point Cost table. Then add 3 points to the
character's abilities, no more than two to any single ability.

\textbf{Ability Score Point Cost}

\begin{longtable}[]{@{}
  >{\raggedright\arraybackslash}p{(\linewidth - 2\tabcolsep) * \real{0.5000}}
  >{\raggedright\arraybackslash}p{(\linewidth - 2\tabcolsep) * \real{0.5000}}@{}}
\toprule\noalign{}
\begin{minipage}[b]{\linewidth}\raggedright
Score
\end{minipage} & \begin{minipage}[b]{\linewidth}\raggedright
Cost
\end{minipage} \\
\midrule\noalign{}
\endhead
\bottomrule\noalign{}
\endlastfoot
8 & 0 \\
9 & 1 \\
10 & 2 \\
11 & 3 \\
12 & 4 \\
13 & 5 \\
14 & 7 \\
15 & 9 \\
\end{longtable}

\subsubsection{Ability
Modifiers}\label{Creating_a_Character_ability-modifiers}

Your final ability scores determine your modifiers. Find your modifier
by subtracting 10 from the ability score and dividing by 2 (round down).

\section{Classes}\label{PlayersGuide_classes}

\textbf{Class Name Changes}

Some class names in Free5e have been updated to remove outdated or
problematic references while staying true to their themes. They use the
same game mechanics as the original classes.

\begin{itemize}
\item
  \textbf{Barbarian -\textgreater{} Dreadnought}\\
  Replacing a derogatory cultural slur
\item
  \textbf{Druid -\textgreater{} Primal}\\
  Removed inaccurate and appropriated portrayal of real-world religion
\item
  \textbf{Monk -\textgreater{} Adept}\\
  Expanding the concept without appropriating a cultural tradition
\item
  \textbf{Paladin -\textgreater{} Vanguard}\\
  Removing the association with a real-world religious conflict\}
\end{itemize}

\subsection{Cleric}\label{Cleric_cleric}

Clerics wield divine power as weapons and shields, serving as living
conduits of their deity's will.

\subsubsection{Class Features}\label{Cleric_class-features}

As a Cleric, you gain the following class features.

\paragraph{Hit Points}\label{Cleric_hit-points}

\begin{itemize}
\item
  \textbf{Hit Dice}\\
  1d8 per Cleric level
\item
  \textbf{Hit Points at 1st Level}\\
  8 + your Constitution modifier
\item
  \textbf{Hit Points at Higher Levels}\\
  1d8 (or 5) + your Constitution modifier per Cleric level after 1st
\end{itemize}

\paragraph{Proficiencies}\label{Cleric_proficiencies}

\begin{itemize}
\item
  \textbf{Armor}\\
  Light armor, medium armor, shields
\item
  \textbf{Weapons}\\
  Simple weapons
\item
  \textbf{Tools}\\
  None
\item
  \textbf{Saving Throws}\\
  Wisdom, Charisma
\item
  \textbf{Skills}\\
  Choose two from History, Insight, Medicine, Persuasion, and Religion
\end{itemize}

\paragraph{Equipment}\label{Cleric_equipment}

You start with the following equipment, in addition to the equipment
granted by your background:

\begin{itemize}
\item
  (a) a mace or (b) a warhammer (if proficient)
\item
  (a) scale mail, (b) leather armor, or (c) chain mail (if proficient)
\item
  (a) a light crossbow and 20 bolts or (b) any simple weapon
\item
  (a) a priest's pack or (b) an explorer's pack
\item
  A shield and a holy symbol
\end{itemize}

\textbf{The Cleric (table)}

\begin{longtable}[]{@{}
  >{\raggedright\arraybackslash}p{(\linewidth - 24\tabcolsep) * \real{0.0769}}
  >{\raggedright\arraybackslash}p{(\linewidth - 24\tabcolsep) * \real{0.0769}}
  >{\raggedright\arraybackslash}p{(\linewidth - 24\tabcolsep) * \real{0.0769}}
  >{\raggedright\arraybackslash}p{(\linewidth - 24\tabcolsep) * \real{0.0769}}
  >{\raggedright\arraybackslash}p{(\linewidth - 24\tabcolsep) * \real{0.0769}}
  >{\raggedright\arraybackslash}p{(\linewidth - 24\tabcolsep) * \real{0.0769}}
  >{\raggedright\arraybackslash}p{(\linewidth - 24\tabcolsep) * \real{0.0769}}
  >{\raggedright\arraybackslash}p{(\linewidth - 24\tabcolsep) * \real{0.0769}}
  >{\raggedright\arraybackslash}p{(\linewidth - 24\tabcolsep) * \real{0.0769}}
  >{\raggedright\arraybackslash}p{(\linewidth - 24\tabcolsep) * \real{0.0769}}
  >{\raggedright\arraybackslash}p{(\linewidth - 24\tabcolsep) * \real{0.0769}}
  >{\raggedright\arraybackslash}p{(\linewidth - 24\tabcolsep) * \real{0.0769}}
  >{\raggedright\arraybackslash}p{(\linewidth - 24\tabcolsep) * \real{0.0769}}@{}}
\toprule\noalign{}
\begin{minipage}[b]{\linewidth}\raggedright
Level
\end{minipage} & \begin{minipage}[b]{\linewidth}\raggedright
Proficiency Bonus
\end{minipage} & \begin{minipage}[b]{\linewidth}\centering
Cantrips Known
\end{minipage} & \begin{minipage}[b]{\linewidth}\centering
1st
\end{minipage} & \begin{minipage}[b]{\linewidth}\centering
2nd
\end{minipage} & \begin{minipage}[b]{\linewidth}\centering
3rd
\end{minipage} & \begin{minipage}[b]{\linewidth}\centering
4th
\end{minipage} & \begin{minipage}[b]{\linewidth}\centering
5th
\end{minipage} & \begin{minipage}[b]{\linewidth}\centering
6th
\end{minipage} & \begin{minipage}[b]{\linewidth}\centering
7th
\end{minipage} & \begin{minipage}[b]{\linewidth}\centering
8th
\end{minipage} & \begin{minipage}[b]{\linewidth}\centering
9th
\end{minipage} & \begin{minipage}[b]{\linewidth}\raggedright
Features
\end{minipage} \\
\midrule\noalign{}
\endhead
\bottomrule\noalign{}
\endlastfoot
1st & +2 & 3 & 2 & - & - & - & - & - & - & - & - &
\hyperref[Cleric_spellcasting]{Spellcasting},
\hyperref[Cleric_divine-domain]{Divine Domain} \\
2nd & +2 & 3 & 3 & - & - & - & - & - & - & - & - &
\hyperref[Cleric_channel-divinity]{Channel Divinity} (1/rest),
\hyperref[Cleric_divine-domain]{Divine Domain} feature \\
3rd & +2 & 3 & 4 & 2 & - & - & - & - & - & - & - & - \\
4th & +2 & 4 & 4 & 3 & - & - & - & - & - & - & - &
\hyperref[Cleric_ability-score-improvement]{Ability Score
Improvement} \\
5th & +3 & 4 & 4 & 3 & 2 & - & - & - & - & - & - &
\hyperref[Cleric_destroy-undead]{Destroy Undead} (CR 1/2) \\
6th & +3 & 4 & 4 & 3 & 3 & - & - & - & - & - & - &
\hyperref[Cleric_channel-divinity]{Channel Divinity} (2/rest),
\hyperref[Cleric_divine-domain]{Divine Domain} feature \\
7th & +3 & 4 & 4 & 3 & 3 & 1 & - & - & - & - & - & - \\
8th & +3 & 4 & 4 & 3 & 3 & 2 & - & - & - & - & - &
\hyperref[Cleric_ability-score-improvement]{Ability Score Improvement},
\hyperref[Cleric_destroy-undead]{Destroy Undead} (CR 1),
\hyperref[Cleric_divine-domain]{Divine Domain} feature \\
9th & +4 & 4 & 4 & 3 & 3 & 3 & 1 & - & - & - & - & - \\
10th & +4 & 5 & 4 & 3 & 3 & 3 & 2 & - & - & - & - &
\hyperref[Cleric_divine-intervention]{Divine Intervention} \\
11th & +4 & 5 & 4 & 3 & 3 & 3 & 2 & 1 & - & - & - &
\hyperref[Cleric_destroy-undead]{Destroy Undead} (CR 2) \\
12th & +4 & 5 & 4 & 3 & 3 & 3 & 2 & 1 & - & - & - &
\hyperref[Cleric_ability-score-improvement]{Ability Score
Improvement} \\
13th & +5 & 5 & 4 & 3 & 3 & 3 & 2 & 1 & 1 & - & - & - \\
14th & +5 & 5 & 4 & 3 & 3 & 3 & 2 & 1 & 1 & - & - &
\hyperref[Cleric_destroy-undead]{Destroy Undead} (CR 3) \\
15th & +5 & 5 & 4 & 3 & 3 & 3 & 2 & 1 & 1 & 1 & - & - \\
16th & +5 & 5 & 4 & 3 & 3 & 3 & 2 & 1 & 1 & 1 & - &
\hyperref[Cleric_ability-score-improvement]{Ability Score
Improvement} \\
17th & +6 & 5 & 4 & 3 & 3 & 3 & 2 & 1 & 1 & 1 & 1 &
\hyperref[Cleric_destroy-undead]{Destroy Undead} (CR 4),
\hyperref[Cleric_divine-domain]{Divine Domain} feature \\
18th & +6 & 5 & 4 & 3 & 3 & 3 & 3 & 1 & 1 & 1 & 1 &
\hyperref[Cleric_channel-divinity]{Channel Divinity} (3/rest) \\
19th & +6 & 5 & 4 & 3 & 3 & 3 & 3 & 2 & 1 & 1 & 1 &
\hyperref[Cleric_ability-score-improvement]{Ability Score
Improvement} \\
20th & +6 & 5 & 4 & 3 & 3 & 3 & 3 & 2 & 2 & 1 & 1 &
\hyperref[Cleric_divine-intervention]{Divine Intervention}
improvement \\
\end{longtable}

\subsubsection{Spellcasting}\label{Cleric_spellcasting}

As a conduit for divine power, you can cast Cleric spells.

\paragraph{Cantrips}\label{Cleric_cantrips}

At 1st level, you know three cantrips of your choice from the
\hyperref[Cleric_Spells_cleric-spells]{Cleric spell list}. You learn
additional Cleric cantrips of your choice at higher levels, as shown in
the Cantrips Known column of the \hyperref[cleric-table]{Cleric table}.

\paragraph{Preparing and Casting
Spells}\label{Cleric_preparing-and-casting-spells}

The \hyperref[cleric-table]{Cleric table} shows how many spell slots you
have to cast your spells of 1st level and higher. To cast one of these
spells, you must expend a slot of the spell's level or higher. You
regain all expended spell slots when you finish a long rest.

You prepare the list of Cleric spells that are available for you to
cast, choosing from the \hyperref[Cleric_Spells_cleric-spells]{Cleric
spell list}. When you do so, choose a number of Cleric spells equal to
your Wisdom modifier + your Cleric level (minimum of one spell). The
spells must be of a level for which you have spell slots.

For example, if you are a 3rd-level Cleric, you have four 1st-level and
two 2nd-level spell slots. With a Wisdom of 16, your list of prepared
spells can include six spells of 1st or 2nd level, in any combination.
If you prepare the 1st-level spell Cure Wounds, you can cast it using a
1st-level or 2nd-level slot. Casting the spell doesn't remove it from
your list of prepared spells.

You can change your list of prepared spells when you finish a long rest.
Preparing a new list of Cleric spells requires time spent in prayer and
meditation: at least 1 minute per spell level for each spell on your
list.

\paragraph{Spellcasting Ability}\label{Cleric_spellcasting-ability}

Wisdom is your spellcasting ability for your Cleric spells. The power of
your spells comes from your devotion to your deity. You use your Wisdom
whenever a Cleric spell refers to your spellcasting ability. In
addition, you use your Wisdom modifier when setting the saving throw DC
for a Cleric spell you cast and when making an attack roll with one.

\begin{itemize}
\item
  \textbf{Spell save DC}\\
  = 8 + your proficiency bonus + your Wisdom modifier
\item
  \textbf{Spell attack modifier}\\
  = your proficiency bonus + your Wisdom modifier
\end{itemize}

\paragraph{Ritual Casting}\label{Cleric_ritual-casting}

You can cast a Cleric spell as a ritual if that spell has the ritual tag
and you have the spell prepared.

\paragraph{Spellcasting Focus}\label{Cleric_spellcasting-focus}

You can use a holy symbol (see chapter Equipment) as a spellcasting
focus for your Cleric spells.

\subsubsection{Divine Domain}\label{Cleric_divine-domain}

Choose one \hyperref[Cleric_domains]{domain related to your deity}. Your
choice grants you domain spells and other features when you choose it at
1st level. It also grants you additional ways to use Channel Divinity
when you gain that feature at 2nd level, and additional benefits at 6th,
8th, and 17th levels.

\paragraph{Domain Spells}\label{Cleric_domain-spells}

Each domain has a list of spells---\hspace{0pt}its domain
spells---\hspace{0pt}that you gain at the Cleric levels noted in the
domain description. Once you gain a domain spell, you always have it
prepared, and it doesn't count against the number of spells you can
prepare each day.

If you have a domain spell that doesn't appear on the Cleric spell list,
the spell is nonetheless a Cleric spell for you.

\subsubsection{Channel Divinity}\label{Cleric_channel-divinity}

At 2nd level, you gain the ability to channel divine energy directly
from your deity, using that energy to fuel magical effects. You start
with two such effects:
\hyperref[Cleric_channel-divinity-turn-undead]{Turn Undead} and an
effect determined by your domain. Some domains grant you additional
effects as you advance in levels, as noted in the domain description.

When you use your Channel Divinity, you choose which effect to create.
You must then finish a short or long rest to use your Channel Divinity
again.

Some Channel Divinity effects require saving throws. When you use such
an effect from this class, the DC equals your Cleric spell save DC.

Beginning at 6th level, you can use your Channel Divinity twice between
rests, and beginning at 18th level, you can use it three times between
rests. When you finish a short or long rest, you regain your expended
uses.

\paragraph{Channel Divinity: Turn
Undead}\label{Cleric_channel-divinity-turn-undead}

As an action, you present your holy symbol and speak a prayer censuring
the undead. Each undead that can see or hear you within 30 feet of you
must make a Wisdom saving throw. If the creature fails its saving throw,
it is turned for 1 minute or until it takes any damage.

A turned creature must spend its turns trying to move as far away from
you as it can, and it can't willingly move to a space within 30 feet of
you. It also can't take reactions. For its action, it can use only the
Dash action or try to escape from an effect that prevents it from
moving. If there's nowhere to move, the creature can use the Dodge
action.

\subsubsection{Ability Score
Improvement}\label{Cleric_ability-score-improvement}

When you reach 4th level, and again at 8th, 12th, 16th, and 19th level,
you can increase one ability score of your choice by 2, or you can
increase two ability scores of your choice by 1. As normal, you can't
increase an ability score above 20 using this feature.

\subsubsection{Destroy Undead}\label{Cleric_destroy-undead}

Starting at 5th level, when an undead fails its saving throw against
your Turn Undead feature, the creature is instantly destroyed if its
challenge rating is at or below a certain threshold, as shown in the
\hyperref[cleric-feature-destroy-undead-table]{Destroy Undead table}.

\textbf{Destroy Undead (table)}

\begin{longtable}[]{@{}
  >{\raggedright\arraybackslash}p{(\linewidth - 2\tabcolsep) * \real{0.5000}}
  >{\raggedright\arraybackslash}p{(\linewidth - 2\tabcolsep) * \real{0.5000}}@{}}
\toprule\noalign{}
\begin{minipage}[b]{\linewidth}\raggedright
Cleric Level
\end{minipage} & \begin{minipage}[b]{\linewidth}\raggedright
Destroys Undead of CR\ldots\hspace{0pt}
\end{minipage} \\
\midrule\noalign{}
\endhead
\bottomrule\noalign{}
\endlastfoot
5th & \$1/2\$ or lower \\
8th & 1 or lower \\
11th & \$2\$ or lower \\
14th & \$3\$ or lower \\
17th & \$4\$ or lower \\
\end{longtable}

\subsubsection{Divine Intervention}\label{Cleric_divine-intervention}

Beginning at 10th level, you can call on your deity to intervene on your
behalf when your need is great.

Imploring your deity's aid requires you to use your action. Describe the
assistance you seek, and roll percentile dice. If you roll a number
equal to or lower than your Cleric level, your deity intervenes. The GM
chooses the nature of the intervention; the effect of any Cleric spell
or Cleric domain spell would be appropriate.

If your deity intervenes, you can't use this feature again for 7 days.
Otherwise, you can use it again after you finish a long rest.

At 20th level, your call for intervention succeeds automatically, no
roll required.

\subsubsection{Domains}\label{Cleric_domains}

\subsubsection{Matyr Domain}\label{Martyr_Domain_matyr-domain}

Followers of gods that believe in taking on the suffering of those
around them. Stalwart and unflinching, these intrepid souls seek to ease
the suffering of those around them. While typically altruistic,
sometimes their motivations are further afield, belonging to strange
cults of suffering or acceptance.

\paragraph{Domain Spells}\label{Martyr_Domain_domain-spells}

You gain domain spells at the Cleric levels listed on the
\hyperref[cleric-domain-martry-spells-table]{Martyr Domain Spells
table}. Once you gain a domain spell, you always have it prepared, and
it doesn't count against the number of spells you can prepare each day.
If you have a domain spell that doesn't appear on the Cleric spell list,
the spell is nonetheless a Cleric spell for you.

\textbf{Martyr Domain Domain Spells (table)}

\begin{longtable}[]{@{}
  >{\raggedright\arraybackslash}p{(\linewidth - 2\tabcolsep) * \real{0.5000}}
  >{\raggedright\arraybackslash}p{(\linewidth - 2\tabcolsep) * \real{0.5000}}@{}}
\toprule\noalign{}
\begin{minipage}[b]{\linewidth}\raggedright
Cleric Level
\end{minipage} & \begin{minipage}[b]{\linewidth}\raggedright
Spells
\end{minipage} \\
\midrule\noalign{}
\endhead
\bottomrule\noalign{}
\endlastfoot
1st & \hyperref[Spell_Cure_Wounds_cure-wounds]{Cure Wounds}, Shield of
Faith \\
3rd & Lesser Restoration, Warding Bond \\
5th & Remove Curse, Revivify \\
7th & Death Ward, Resilient Sphere \\
9th & Greater Restoration, Mass Cure Wounds \\
\end{longtable}

\paragraph{Clad in Grace}\label{Martyr_Domain_clad-in-grace}

Starting at 1st level, while you are not wearing any armor, your Armor
Class equals 10 + your Constitution modifier + your Wisdom modifier. You
can use a shield and still gain this benefit.

\paragraph{Relieve Suffering}\label{Martyr_Domain_relieve-suffering}

Starting at 1st level, when you cast a spell of 1st level or higher that
restores hit points, you can expend some of your own hit points to
increase the amount another creature that is healed by the spell
regains, expending a number of hit points up to your Cleric level to
increase the amount one creature is healed by an equivalent value.

Additionally, you can touch a creature suffering a disease or the
blinded, deafened, paralyzed or poisoned condition and transfer that
disease or condition to yourself. Starting at 9th level, the range of
conditions you can take increases, including exhaustion (1 level at a
time), petrified, stunned, the effect of a curse, or any reduction to
their ability scores. You are afflicted by the effect in the same manner
the creature you took it was, for the same duration and possible methods
of ending the effect.

You can transfer a condition affecting another creature to yourself a
number of times equal to your Wisdom modifier, regaining all uses on a
long rest.

\paragraph{Channel Divinity: Bear the
Burden}\label{Martyr_Domain_channel-divinity-bear-the-burden}

Starting at 2nd level, you can use your
\hyperref[Cleric_channel-divinity]{Channel Divinity} to bring all the
suffering your allies suffer onto yourself.

As an action, you present your holy symbol and enter a divine trance.
Until the start of your next turn, your movement speed becomes zero and
you gain temporary hit points equal to twice your Cleric level. Any time
an allied creature within 30 feet of you takes damage, they have
resistance to the damage taken, but you take damage equal to the damage
they take.

\paragraph{Overcome Adversity}\label{Martyr_Domain_overcome-adversity}

Starting at 6th level, when you use your
\hyperref[Martyr_Domain_relieve-suffering]{Relieve Suffering} to take
the condition affecting a creature, you can repeat the original saving
throw against the feature if it had one. On success, the condition ends
instead of being transferred.

Additionally, when you take damage as a result of the Warding Bond
spell, you have resistance to that damage.

\paragraph{Divine Empowerment}\label{Martyr_Domain_divine-empowerment}

At 8th level, pick one of the following options:

\begin{itemize}
\item
  \textbf{Divine Strike}\\
  At 8th level, you gain the ability to infuse your weapon strikes with
  divine energy. Once on each of your turns when you hit a creature with
  a weapon attack made as part of the attack action, you can cause the
  attack to deal an extra 1d8 radiant damage to the target.

  When you reach 11th level, the extra damage increases to 2d8, and when
  you reach 17th level, the extra damage increases to 3d8.
\item
  \textbf{Potent Spellcasting}\\
  Starting at 8th level, you add your Wisdom modifier to the damage you
  deal with any Cleric cantrip.
\end{itemize}

\paragraph{Unyielding
Concentration}\label{Martyr_Domain_unyielding-concentration}

Starting at 17th level, while you are concentrating on a Cleric spell
that targets you or your allies, your concentration can't be broken as a
result of taking damage.

\subsubsection{Sacred Mandate}\label{Sacred_Mandate_sacred-mandate}

Your deity doesn't assign you a job---\hspace{0pt}they grant you a
mandate.

You embody a divine imperative---\hspace{0pt}an expression of your
deity's focus. This Sacred Mandate grants you access to specific powers
that you shape through your chosen domain.

At 1st level, you choose a \textbf{\emph{Mandate Domain}}, such as
Knowledge, Life, or War. Your mandate grants you a collection of
features as outlined below.

\paragraph{Mandate Spells}\label{Sacred_Mandate_mandate-spells}

You gain the spell list associated with your selected domain. These
spells count as Cleric spells for you and are always prepared.

\textbf{The Mandate Spells (table)}

\begin{longtable}[]{@{}
  >{\raggedright\arraybackslash}p{(\linewidth - 10\tabcolsep) * \real{0.1667}}
  >{\raggedright\arraybackslash}p{(\linewidth - 10\tabcolsep) * \real{0.1667}}
  >{\raggedright\arraybackslash}p{(\linewidth - 10\tabcolsep) * \real{0.1667}}
  >{\raggedright\arraybackslash}p{(\linewidth - 10\tabcolsep) * \real{0.1667}}
  >{\raggedright\arraybackslash}p{(\linewidth - 10\tabcolsep) * \real{0.1667}}
  >{\raggedright\arraybackslash}p{(\linewidth - 10\tabcolsep) * \real{0.1667}}@{}}
\toprule\noalign{}
\begin{minipage}[b]{\linewidth}\raggedright
Domain
\end{minipage} & \begin{minipage}[b]{\linewidth}\raggedright
1st Level
\end{minipage} & \begin{minipage}[b]{\linewidth}\raggedright
3rd Level
\end{minipage} & \begin{minipage}[b]{\linewidth}\raggedright
5th Level
\end{minipage} & \begin{minipage}[b]{\linewidth}\raggedright
7th Level
\end{minipage} & \begin{minipage}[b]{\linewidth}\raggedright
9th Level
\end{minipage} \\
\midrule\noalign{}
\endhead
\bottomrule\noalign{}
\endlastfoot
Knowledge & Identify, Command & Augury, Detect Thoughts & Clairvoyance,
Speak with Dead & Arcane Eye, Divination & Legend Lore, Scrying \\
Life & \hyperref[Spell_Cure_Wounds_cure-wounds]{Cure Wounds},
\hyperref[Spell_Bless_bless]{Bless} & Lesser Restoration, Spiritual
Weapon & Beacon of Hope, Revivify & Death Ward, Guardian of Faith & Mass
Cure Wounds, Raise Dead \\
Light & Color Spray, Guiding Bolt & Continual Flame, Moonbeam &
Daylight, Spirit Guardians & Arcane Eye, Divination & Dream, Scrying \\
Storm & Thunderwave, Fog Cloud & Gust of Wind, Shatter & Call Lightning,
Wind Wall & Control Water, Ice Storm & Control Weather, Flame Strike \\
Trickery & Disguise Self, Charm Person & Mirror Image, Pass Without
Trace & Bestow Curse, Major Image & Greater Invisibility, Hallucinatory
Terrain & Mislead, Seeming \\
War & Divine Favor, Shield of Faith & Magic Weapon, Spiritual Weapon &
Revivify, Slow & Stoneskin, Freedom of Movement & Hold Monster, Flame
Strike \\
\end{longtable}

\paragraph{Mandate
Proficiency}\label{Sacred_Mandate_mandate-proficiency}

When you choose this domain at 1st level, choose one additional
proficiency from a list thematic to your mandate:

\begin{itemize}
\item
  \textbf{Knowledge}\\
  One language, Arcana or History
\item
  \textbf{Life}\\
  Medicine or Herbalism Kit
\item
  \textbf{Light}\\
  Performance or Persuasion
\item
  \textbf{Storm}\\
  Athletics or Nature
\item
  \textbf{Trickery}\\
  Deception or Sleight of Hand
\item
  \textbf{War}\\
  Martial weapon or Heavy armor
\end{itemize}

\paragraph{Divine Gift}\label{Sacred_Mandate_divine-gift}

Starting at 1st level, you gain a unique gift granted by your divine
mandate. You may use this feature a number of times equal to your Wisdom
modifier (minimum of once) per long rest.

\begin{itemize}
\item
  \textbf{Guided Smite (War, Storm, Light, Trickery)}\\
  When you take the Attack action, you may make an additional weapon
  attack as a bonus action. This attack deals extra radiant, thunder,
  lightning, or psychic damage (choose one).

  When you reach 10th Level, on a critical hit, you can make another
  weapon attack as part of the same action.
\item
  \textbf{Protective Intercession (Life, Light, Knowledge)}\\
  When a creature within 30 feet takes damage, you can use your reaction
  to reduce the damage by 1d10 + your Wisdom modifier.

  When you reach 10th Level, the creature gains temporary hit points
  equal to your Cleric level.
\item
  \textbf{Insightful Surge (Knowledge, Trickery)}\\
  As a bonus action, you grant a creature that can see or hear you
  within 30 feet advantage on their next attack roll, ability check, or
  saving throw.

  When you reach 10th Level, when using this ability, you also give a
  creature within range that can see or hear you, disadvantage on its
  next attack roll or ability check as you momentarily distract it or
  otherwise exploit a subtle weakness.
\end{itemize}

\paragraph{Channel Divinity}\label{Sacred_Mandate_channel-divinity}

Starting at 2nd level, you gain one of the following Channel Divinity
options.

When you reach 6th level, the effects improve or gain additional
features.

\begin{itemize}
\item
  \textbf{Channel Divinity: War Cry (War, Storm)}\\
  As a bonus action, choose a number of creatures equal to your Wisdom
  modifier (minimum 1) that you can see within 30 feet. Each target can
  immediately use their reaction to move up to half their speed without
  provoking an opportunity attack and make one weapon attack.

  When you reach 6th Level, targets add your Cleric level to their
  damage roll for this attack.
\item
  \textbf{Channel Divinity: Luminous Infusion (Light, Storm)}\\
  As an action, you cause divine light to flare in a 15-foot radius
  centered on yourself. Choose one damage type: radiant, lightning, or
  fire. Until the start of your next turn, any creature that enters or
  ends its turn in the area or hits you with a melee attack takes 3
  damage. The damage increases by 2 each time your proficiency bonus
  increases.

  When you reach 6th Level, allies in the aura deal +1d4 + your
  proficiency bonus damage of the chosen type on weapon attacks.
\item
  \textbf{Channel Divinity: Revelatory Surge (Knowledge, Life,
  Trickery)}\\
  You call forth a flash of divine understanding as an action that
  momentarily reveals hidden truths. For 1 minute, you and allies in the
  area gain advantage on one Intelligence, Wisdom, or Charisma check of
  your choice, as divine clarity sharpens your senses and mind.

  When you reach 6th Level, you and each creature of your choice within
  30 feet have truesight for one minute.
\item
  \textbf{Channel Divinity: Phantom's Whim (Trickery, Knowledge)}\\
  As a bonus action, you cloak a creature that you choose within 30 feet
  in illusions. Until the start of your next turn, each target has
  partial cover and can take the Hide action as a bonus action.

  When you reach 6th Level, the number of affected creatures increases
  by your proficiency bonus.
\item
  \textbf{Channel Divinity: Aegis of Faith (Life, War)}\\
  As an action, you conjure a divine shield around an ally within 30
  feet. Until the end of their next turn, the target adds your
  proficiency bonus to AC or resistance to necrotic or radiant damage.

  When you reach 6th Level, the target also gains temporary hit points
  equal to 2d6 + your Wisdom modifier when you use this feature.
\item
  \textbf{Channel Divinity: Echo of Command (Trickery, War)}\\
  As a bonus action, choose a creature whose location you know within 30
  feet.

  If the creature is an ally, it can immediately use its reaction to
  make one weapon attack or cast a cantrip. It adds your Wisdom modifier
  to the attack roll.

  If the creature is an enemy, it must make a Wisdom saving throw. On a
  failure, it uses its reaction to make a weapon attack or a spell
  attack with a cantrip, targeting a creature you choose.

  When you reach 6th Level, when you use this ability, you may choose
  whether affected creatures can be frightened or charmed until the end
  of their next turn.
\end{itemize}

\paragraph{Divine Strike or Potent
Spellcasting}\label{Sacred_Mandate_divine-strike-or-potent-spellcasting}

When you reach 8th level, choose one of the following:

\begin{itemize}
\item
  \textbf{Divine Strike.}\\
  Your weapon attacks deal an extra 1d8 damage of a type appropriate to
  your mandate, increasing to 2d8 at 14th level.

  Suggested Damage Types:

  \begin{itemize}
  \item
    \textbf{Knowledge}\\
    Psychic
  \item
    \textbf{Life}\\
    Radiant
  \item
    \textbf{Light}\\
    Radiant
  \item
    \textbf{Storm}\\
    Lightning or Thunder (choose one)
  \item
    \textbf{Trickery}\\
    Poison or Acid (choose one)
  \item
    \textbf{War}\\
    Bludgeoning, Piercing, or Slashing (choose one)
  \end{itemize}
\item
  \textbf{Potent Spellcasting.}\\
  You add your Wisdom modifier to the damage you deal with Cleric
  cantrips.
\end{itemize}

\paragraph{Exalted Mandate}\label{Sacred_Mandate_exalted-mandate}

At 17th level, choose one matching your mandate. You can use this
feature once per long rest.

\begin{itemize}
\item
  \textbf{Avatar of War (War, Storm)}\\
  When reduced to 0 hit points, you drop to 1 hit pointinstead and may
  use your reaction to make one weapon attack. You also gain resistance
  to bludgeoning, piercing, and slashing damage for 1 minute.
\item
  \textbf{Illuminated Soul (Light, Life)}\\
  As an action, you emit a 30 foot aura for 1 minute. Allies in the area
  gain advantage on saving throws against being blinded, charmed, or
  frightened and regain 1d6 HP at the start of their turns.
\item
  \textbf{Shifting Facade (Trickery, Knowledge)}\\
  The locations of anyone within 30 feet of you become difficult for
  enemies to determine for one minute, as all creatures that you choose,
  including you, seem to switch places with each other. Enemies must
  succeed on a Wisdom saving throw or target a random creature when
  trying to attack any creature within range.
\end{itemize}

\section{Playing the Game}\label{PlayersGuide_playing-the-game}

\subsection{Using Ability
Scores}\label{Using_Ability_Scores_using-ability-scores}

Six abilities provide a quick description of every creature's physical
and mental characteristics:

\begin{itemize}
\item
  \textbf{Strength,}\\
  measuring physical power
\item
  \textbf{Dexterity,}\\
  measuring agility
\item
  \textbf{Constitution,}\\
  measuring endurance
\item
  \textbf{Intelligence,}\\
  measuring reasoning and memory
\item
  \textbf{Wisdom,}\\
  measuring perception and insight
\item
  \textbf{Charisma,}\\
  measuring force of personality
\end{itemize}

Is a character muscle-bound and insightful? Brilliant and charming?
Nimble and hardy? Ability scores define these qualities---\hspace{0pt}a
creature's assets as well as weaknesses.

The three main rolls of the game---\hspace{0pt}the ability check, the
saving throw, and the attack roll---\hspace{0pt}rely on the six ability
scores. The book's introduction describes the basic rule behind these
rolls: roll a d20, add an ability modifier derived from one of the six
ability scores, and compare the total to a target number.

\section{Rules of Magic}\label{PlayersGuide_rules-of-magic}

\subsection{Spellcasting}\label{About_Spellcasting_spellcasting}

Magic rules fantasy gaming worlds, often as spells. This section covers
spellcasting. Character classes and monsters have unique spell learning
and preparation methods. Regardless of origin, spells follow these
rules.

\subsubsection{What Is a
Spell?}\label{About_Spellcasting_what-is-a-spell}

A spell is a discrete magical effect, a single shaping of magical
energies in the multiverse. Casting a spell involves plucking, pinning,
vibrating, and releasing invisible strands of magic to unleash the
desired effect, usually in seconds. Spells can be versatile tools,
weapons, or protective wards, dealing damage, undoing it, imposing or
removing conditions, draining life energy, or restoring it. Thousands of
spells have been created throughout history, many forgotten. Some might
be recorded in ancient spellbooks or trapped in the minds of dead gods,
while others could be reinvented by powerful characters.

\subsubsection{Concentration}\label{About_Spellcasting_concentration}

Some spells require concentration to maintain their magic. If you lose
concentration, the spell ends. If a spell needs concentration, its
Duration entry specifies how long you can concentrate. You can end
concentration anytime. Normal activity doesn't interfere. The following
can break concentration:

\begin{itemize}
\item
  \textbf{Casting another spell that requires concentration.}\\
  You lose concentration on a spell if you cast another spell that
  requires concentration. You can't concentrate on two spells at once.
\item
  \textbf{Taking damage.}\\
  Whenever you take damage while concentrating on a spell, make a
  Constitution saving throw to maintain concentration. The DC is 10 or
  half the damage, whichever is higher. If you take damage from multiple
  sources, make a separate saving throw for each.
\item
  \textbf{Being incapacitated or killed.}\\
  You lose concentration on a spell if incapacitated or die. The GM may
  also require a DC 10 Constitution saving throw to maintain
  concentration on a spell due to certain environmental phenomena, like
  a crashing wave on a storm-tossed ship.
\end{itemize}

\subsection{Spell Lists}\label{PlayersGuide_spell-lists}

\subsubsection{Cleric Spells}\label{Cleric_Spells_cleric-spells}

\paragraph{Cantrips (0 Level)}\label{Cleric_Spells_cantrips-0-level}

\begin{itemize}
\item
  Guidance
\item
  Mending
\item
  Sacred Flame
\end{itemize}

\paragraph{1st Level}\label{Cleric_Spells_1st-level}

\begin{itemize}
\item
  Bane
\item
  Ceremony
\item
  Command
\item
  \hyperref[Spell_Cure_Wounds_cure-wounds]{Cure Wounds}
\item
  Detect Magic
\item
  Guiding Bolt
\item
  Inflict Wounds
\item
  Protection from Evil and Good
\item
  Purify Food and Drink
\item
  Sanctuary
\item
  Shield of Faith
\end{itemize}

\paragraph{2nd Level}\label{Cleric_Spells_2nd-level}

\begin{itemize}
\item
  \hyperref[Spell_Aid_aid]{Aid}
\item
  Augury
\item
  Blindness/Deafness
\item
  Calm Emotions
\item
  Continual Flame
\item
  Enhance Ability
\item
  Find Traps
\item
  Gentle Repose
\item
  Hold Person
\item
  Lesser Restoration
\item
  Locate Object
\item
  Prayer of Healing
\item
  Protection from Poison
\item
  Silence
\item
  Spiritual Weapon
\item
  Warding Bond
\item
  Zone of Truth
\end{itemize}

\paragraph{3rd Level}\label{Cleric_Spells_3rd-level}

\begin{itemize}
\item
  \hyperref[Spell_Animate_Dead_animate-dead]{Animate Dead}
\item
  Beacon of Hope
\item
  Bestow Curse
\item
  Clairvoyance
\item
  Create Food and Water
\item
  Daylight
\item
  Dispel Magic
\item
  Glyph of Warding
\item
  Magic Circle
\item
  Mass Healing Word
\item
  Meld into Stone
\item
  Protection from Energy
\item
  Remove Curse
\item
  Revivify
\item
  Sending
\item
  Speak with Dead
\item
  Spirit Guardians
\item
  Tongues
\item
  Water Walk
\end{itemize}

\paragraph{4th Level}\label{Cleric_Spells_4th-level}

\begin{itemize}
\item
  Banishment
\item
  Control Water
\item
  Death Ward
\item
  Divination
\item
  Freedom of Movement
\item
  Guardian of Faith
\item
  Locate Creature
\end{itemize}

\paragraph{5th Level}\label{Cleric_Spells_5th-level}

\begin{itemize}
\item
  Commune
\item
  Dispel Evil and Good
\item
  Geas
\item
  Hallow
\item
  Legend Lore
\item
  Planar Binding Raise Dead Scrying
\end{itemize}

\paragraph{6th Level}\label{Cleric_Spells_6th-level}

\begin{itemize}
\item
  Blade Barrier
\item
  Find the Path
\item
  Harm
\item
  Heroes' Feast
\item
  True Seeing
\end{itemize}

\paragraph{7th Level}\label{Cleric_Spells_7th-level}

\begin{itemize}
\item
  Conjure Celestial
\item
  Etherealness
\item
  Plane Shift
\item
  Resurrection
\end{itemize}

\paragraph{8th Level}\label{Cleric_Spells_8th-level}

\begin{itemize}
\item
  Antimagic Field
\item
  Earthquake
\end{itemize}

\paragraph{9th Level}\label{Cleric_Spells_9th-level}

\begin{itemize}
\item
  Astral Projection
\item
  Mass Heal True Resurrection
\end{itemize}

\paragraph{Spell
Descriptions}\label{All_Spell_Descriptions_spell-descriptions}

\subparagraph{Acid Splash}\label{Spell_Acid_Splash_acid-splash}

\emph{Conjuration cantrip}

\begin{itemize}
\item
  \textbf{Casting Time}\\
  1 action
\item
  \textbf{Range}\\
  60 feet
\item
  \textbf{Target}\\
  Up to two creatures within 5 feet of each other
\item
  \textbf{Components}\\
  V, S
\item
  \textbf{Duration}\\
  Instantaneous
\item
  \textbf{Saving Throw}\\
  Dexterity negates
\end{itemize}

A stinking bubble of acid is conjured out of thin air to fly at the
targets, dealing 1d6 acid damage.

This spell's damage increases by 1d6 when you reach 5th level (2d6),
11th level (3d6), and 17th level (4d6).

\subparagraph{Aid}\label{Spell_Aid_aid}

\emph{2nd-level abjuration}

\begin{itemize}
\item
  \textbf{Casting Time}\\
  1 action
\item
  \textbf{Range}\\
  60 feet
\item
  \textbf{Target}\\
  Up to three creatures
\item
  \textbf{Components}\\
  V, S, M (measure of spirits)
\item
  \textbf{Duration}\\
  8 hours
\end{itemize}

You draw upon divine power, imbuing the targets with fortitude. Until
the spell ends, each target increases its hit point maximum and current
hit points by 5.

\begin{itemize}
\item
  \textbf{At Higher Levels.}\\
  The granted hit points increase by an additional 5 for each slot level
  above 2nd.
\end{itemize}

\subparagraph{Animate Dead}\label{Spell_Animate_Dead_animate-dead}

\emph{3rd-level necromancy}

\begin{itemize}
\item
  \textbf{Casting Time}\\
  1 minute
\item
  \textbf{Range}\\
  Touch
\item
  \textbf{Components}\\
  V, S, M (two copper coins)
\item
  \textbf{Duration}\\
  Instantaneous
\end{itemize}

You animate a mortal's remains to become your undead servant.

If the spell is cast upon bones you create a skeleton, and if cast upon
a corpse you can choose to create a skeleton or a zombie. The GM has the
undead's statistics.

While it is within 60 feet you can use a bonus action to mentally
command any undead you created with this spell. When you command
multiple undead using this spell, you must give them all the same
command. You may decide the creature's exact action and move, or you can
issue a general command, such as guarding an area, which it follows
until the task is complete or you issue it a new command. If not given a
command, the undead only defends itself.

The undead is under your control for 24 hours unless you cast this spell
on it before the spell ends to maintain control of it for another 24
hours. Casting the spell in this way reasserts control over up to 4 of
your previously-animated undead instead of animating a new one. When no
longer under your control, the undead no longer obeys your commands.

\begin{itemize}
\item
  \textbf{At Higher Levels.}\\
  You create or maintain control over 2 additional undead for each slot
  level above 3rd.
\end{itemize}

\subparagraph{Animate
Objects}\label{Spell_Animate_Objects_animate-objects}

\emph{5th-level transmutation}

\begin{itemize}
\item
  \textbf{Casting Time}\\
  1 action
\item
  \textbf{Range}\\
  Long (120 feet)
\item
  \textbf{Components}\\
  V, S
\item
  \textbf{Duration}\\
  Concentration (1 minute)
\end{itemize}

Choose up to 6 unattended nonmagical Small or Tiny objects. You may also
choose larger objects; treat Medium objects as 2 objects, Large objects
as 3 objects, and Huge objects as 6 objects.

Until the spell ends or a target is reduced to 0 hit points, you animate
the targets and turn them into constructs under your control.

Each construct has Constitution 10, Intelligence 3, Wisdom 3, and
Charisma 1, as well as a flying speed of 30 feet and the ability to
hover (if securely fastened to something larger, it has a Speed of 0),
and blindsight to a range of 30 feet (blind beyond that distance).
Otherwise a construct's statistics are determined by its size.

If you animate 4 or more Small or Tiny objects, instead of controlling
each construct individually they function as a construct swarm. Add
together all swarm's total hit points. Attacks against a construct swarm
deal half damage. The construct swarm reverts to individual constructs
when it is reduced to 15 hit points or less.

You can use a bonus action to mentally command any construct made with
this spell while within 500 feet of it. When you command multiple
constructs using this spell, you must give them all the same command.
You may decide the creature's exact action and move, or you can issue a
general command, such as guarding an area, which it follows until the
task is complete or you issue it a new command. If not given a command,
the construct only defends itself.

When you command a construct to attack, it makes a slam, a melee attack,
against a creature within 5 feet of it. On a hit the construct deals
bludgeoning, piercing, or slashing damage appropriate to its shape.

When the construct drops to 0 hit points, any excess damage carries over
to its inanimate object form.

\begin{itemize}
\item
  \textbf{At Higher Levels.}\\
  You can animate 2 additional Small or Tiny objects for each slot level
  above 5th.
\end{itemize}

\textbf{Animate Objects Table}

\begin{longtable}[]{@{}
  >{\raggedright\arraybackslash}p{(\linewidth - 10\tabcolsep) * \real{0.1667}}
  >{\raggedright\arraybackslash}p{(\linewidth - 10\tabcolsep) * \real{0.1667}}
  >{\raggedright\arraybackslash}p{(\linewidth - 10\tabcolsep) * \real{0.1667}}
  >{\raggedright\arraybackslash}p{(\linewidth - 10\tabcolsep) * \real{0.1667}}
  >{\raggedright\arraybackslash}p{(\linewidth - 10\tabcolsep) * \real{0.1667}}
  >{\raggedright\arraybackslash}p{(\linewidth - 10\tabcolsep) * \real{0.1667}}@{}}
\toprule\noalign{}
\begin{minipage}[b]{\linewidth}\raggedright
Size
\end{minipage} & \begin{minipage}[b]{\linewidth}\centering
HP
\end{minipage} & \begin{minipage}[b]{\linewidth}\centering
AC
\end{minipage} & \begin{minipage}[b]{\linewidth}\raggedright
Attack
\end{minipage} & \begin{minipage}[b]{\linewidth}\centering
STR
\end{minipage} & \begin{minipage}[b]{\linewidth}\centering
DEX
\end{minipage} \\
\midrule\noalign{}
\endhead
\bottomrule\noalign{}
\endlastfoot
Tiny & 5 & 14 & +6 to hit, 1d4 --- 3 damage & 4 & 18 \\
Small & 10 & 12 & +4 to hit, 1d6 --- 2 damage & 6 & 14 \\
Swarm of Tiny and Small & varies & 13 & +5 to hit, 2d6 damage (1d6
damage if bloodied) & 5 & 16 \\
Medium & 20 & 11 & +3 to hit, 1d8 damage & 10 & 12 \\
Large & 40 & 10 & +4 to hit, 2d8 + 2 damage & 14 & 10 \\
Huge & 80 & 8 & +6 to hit, 2d12 + 4 damage & 18 & 6 \\
\end{longtable}

\subparagraph{Bless}\label{Spell_Bless_bless}

\emph{1st-level enchantment}

\begin{itemize}
\item
  \textbf{Casting Time}\\
  1 action
\item
  \textbf{Range}\\
  30 feet
\item
  \textbf{Components}\\
  V, S, M (a sprinkling of holy water)
\item
  \textbf{Duration}\\
  Concentration, up to 1 minute
\end{itemize}

Until the spell ends, a d4 is added to attack rolls and saving throws
made by a target.

\begin{itemize}
\item
  \textbf{At Higher Levels.}\\
  You target one additional creature for each slot level above 1st.
\end{itemize}

\subparagraph{Cure Wounds}\label{Spell_Cure_Wounds_cure-wounds}

\emph{1st-level evocation}

\begin{itemize}
\item
  \textbf{Casting Time}\\
  1 action
\item
  \textbf{Range}\\
  Touch
\item
  \textbf{Components}\\
  V, S
\item
  \textbf{Duration}\\
  Instantaneous
\end{itemize}

One creature that is neither a construct nor undead regains hit points
equal to 1d8 + your spellcasting ability modifier.

\begin{itemize}
\item
  \textbf{At Higher Levels.}\\
  The hit points regained increase by 1d8 for each slot level above 1st.
\end{itemize}

\section{Appendix}\label{PlayersGuide_appendix}

\subsection{Credits}\label{Credits_credits}

\begin{itemize}
\item
  \textbf{Project Lead \& Writer}\\
  Dale Critchley \url{https://wyrmworkspublishing.com}
\item
  \textbf{Collaborator, Writer, \& Editor}\\
  PJ Coffey \url{https://homebrewandhacking.com/}
\item
  \textbf{Layout \& Accessibility}\\
  Chris Hopper \url{https://bio.link/chrishopper}
\end{itemize}

\subsubsection{Artists}\label{Credits_artists}

\subsubsection{Primary Sources}\label{Credits_primary-sources}

This project was made possible by Sly Flourish, EN Publishing,
KibblesTasty, and Wizards of the Coast, who generously made content
available under a Creative Commons license. We are deeply grateful for
their contributions.

\subsection{Legal}\label{Legal_legal}

This work includes material taken from the A5E System Reference Document
(A5ESRD) by EN Publishing and available at A5ESRD.com, based on Level
Up: Advanced 5th Edition, available at www.levelup5e.com{[}{]}. The
A5ESRD is licensed under the Creative Commons Attribution 4.0
International License available at
\url{https://creativecommons.org/licenses/by/4.0/legalcode}.

This work includes material taken from the System Reference Document 5.1
(``SRD 5.1'') by Wizards of the Coast LLC and available at
\url{https://dnd.Wizards.com/resources/systems-reference-document}. The
SRD 5.1 is licensed under the Creative Commons Attribution 4.0
International License available at
\url{https://creativecommons.org/licenses/by/4.0/legalcode}.

This work includes material taken from the Lazy GM's Resource Document
by Michael E. Shea of SlyFlourish.com, available under a Creative
Commons Attribution 4.0 International License.

This work includes content from Kibbles' Compendium of Legends and
Legacies by KibblesTasty Homebrew LLC and available at
\url{https://www.kthomebrew.com/krd}. The Kibbles' Compendium of Legends
and Legacies is licensed under the Creative Commons Attribution 4.0
International License (CC-BY-4.0) available at
\url{https://creativecommons.org/licenses/by/4.0/legalcode}.

The text of Free5e Player's Guide © 2025 by Wyrmworks Publishing and
available at \url{https://free5e.com} is licensed under Creative Commons
Attribution 4.0 International. To view a copy of this license, visit
\url{https://creativecommons.org/licenses/by/4.0/} You are free to use
this content in any manner permitted by that license as long as you
include the following attribution statement in your own work:

This work includes material adapted from the \textbf{\emph{Free5e
Player's Guide}}, © 2025 by Wyrmworks Publishing, and available at
\url{https://free5e.com}. The Free5e Player's Guide is licensed under
the Creative Commons Attribution 4.0 International License (CC-BY-4.0).
To view a copy of this license, visit
\url{https://creativecommons.org/licenses/by/4.0/}.

This adaptation also includes material originally taken from:

\begin{itemize}
\item
  The \textbf{\emph{A5E System Reference Document (A5ESRD)}} by EN
  Publishing, available at A5ESRD.com and licensed under CC-BY-4.0.
\item
  The \textbf{\emph{System Reference Document 5.1 (SRD 5.1)}} by Wizards
  of the Coast LLC, available at
  \url{https://dnd.Wizards.com/resources/systems-reference-document} and
  licensed under CC-BY-4.0.
\item
  The \textbf{\emph{Lazy GM's Resource Document}} by Michael E. Shea of
  SlyFlourish.com, licensed under CC-BY-4.0.
\item
  The \textbf{\emph{Kibbles' Compendium of Legends and Legacies}} by
  KibblesTasty Homebrew LLC, available at
  \url{https://www.kthomebrew.com/krd} and licensed under CC-BY-4.0.
\end{itemize}

All artwork contained in this book is licensed under CC-BY-4.0 or CC-0.
To use that artwork outside of this project, see the accompanying
document with specific licensing and credit information. If you received
this copy without that document, you can obtain a copy at
\url{https://free5e.com}.
